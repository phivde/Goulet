%%% Copyright (C) 2019 Vincent Goulet
%%%
%%% Ce fichier fait partie du projet
%%% «Programmer avec R»
%%% https://gitlab.com/vigou3/programmer-avec-r
%%%
%%% Cette création est mise à disposition sous licence
%%% Attribution-Partage dans les mêmes conditions 4.0
%%% International de Creative Commons.
%%% https://creativecommons.org/licenses/by-sa/4.0/

\chapter{Algorithmes et algorithmique}
\label{chap:algorithmes}

\def\scriptfilename{\currfilebase.R}

\begin{objectifs}
\item Expliquer les algorithmes classiques de tri et de recherche;
  comparer leurs avantages et inconvénients.
\item Exprimer une méthode de résolution de problème sous la forme
  d'un algorithme à l'aide de pseudo-code.
\item Effectuer la mise en oeuvre d'un algorithme donné dans le
  langage R.
\end{objectifs}

Marianne et Alexandre doivent fournir une solution logicielle à un
problème donné. Sitôt la lecture de l'énoncé du problème complétée,
ils se précipitent sur le clavier pour coder leur solution. Rapidement
bloqués, pensant qu'ils butent sur un problème de syntaxe ou
d'utilisation d'une fonction, ils demandent de l'aide. Lorsqu'on leur
demande qu'est-ce qu'ils essaient de faire, quelle est la nature de leur
solution, ils ne peuvent répondre que par un long silence.

Dans mon rôle de professeur, j'ai été témoin de la scène ci-dessus à
de nombreuses reprises. Marianne et Alexandre font une erreur commune
chez les programmeurs --- et pas que chez les débutants: commencer à
coder sans avoir au préalable suffisamment réfléchi à la solution. Il
manque un plan clair des étapes à suivre pour résoudre le problème.
Or, ce plan, ou la recette à suivre --- nous reviendrons sur cette
analogie culinaire dans la suite --- pour transformer les entrants
d'un problème en une solution, c'est l'algorithme. Travailler sans
algorithme, c'est comme naviguer sans boussole.

Si programmer requiert des algorithmes, l'étude de la programmation
doit donc nécessairement s'accompagner de notions
d'\index{algorithmique}algorithmique, la science qui étudie les
algorithmes et les structures de données. L'algorithmique est une
discipline riche en techniques ingénieuses et en analyses
mathématiques poussées. Connaitre ses principes de base et étudier
certains algorithmes classiques vous permettra de mieux planifier vos
méthodes de résolution de problème. En effet, un bon algorithme permet
de résoudre en quelques secondes un problème qui pourrait autrement
prendre des années.

L'étude des algorithmes s'accompagne habituellement de celle des
structures de données ou, en d'autres termes, de la manière
d'organiser les données dans un ordinateur. Comme nous le verrons au
chapitre \autoref{chap:donnees}, il existe bien différentes structures
de données en R, mais leur réelle mise en œuvre demeure tout à fait
transparente pour les programmeuses et les programmeurs. C'est
pourquoi nous ferons l'impasse sur des notions que l'on retrouve dans
tous les ouvrages classiques d'algorithmique, comme le tableau
(\emph{array}), la liste chainée (\emph{list}), l'arbre (\emph{tree})
ou la table de hachage (\emph{hashtable}).

Le présent chapitre est inspiré de l'excellent ouvrage \emph{Essential
  Algorithms} de \citet{Stephens:algorithms:2013}.


\subsection{Algorithme}
\label{sec:informatique:concepts:algorithme}

Vous avez probablement déjà trié une liste de mots à l'aide d'un
ordinateur. Pourtant, l'ordinateur n'a aucune connaissance du concept
d'ordre alphabétique. Et même s'il «connaissait» cet ordre, comment au
juste réalise-t-il l'opération de tri? Ce genre de questions relève
d'une pierre d'assise de l'informatique:
l'\index{algorithmique}algorithmique, la science qui étudie les
algorithmes et les structures de données.

Un \index{algorithme}\emph{algorithme} est une procédure de calcul
permettant de résoudre un problème bien spécifié. En cela,
l'algorithme explique comment, à partir d'entrants, obtenir l'extrant
solution du problème. Quand on y pense, ce n'est pas très différent
d'une recette culinaire où les entrants sont les ingrédients, la
procédure les étapes de la recette et les extrants, le mets.

À titre d'exemple, supposons que vous devez calculer, pour un jeu de
données quelconque, l'écart moyen des données supérieures à $12$ par
rapport à cette valeur. Un premier algorithme de très haut niveau
permettant de résoudre ce problème irait comme suit.

\begin{Schunk}
  \begin{enumerate}
  \item Sélectionner les données supérieures à $12$.
  \item Soustraire $12$ de chaque valeur retenue à la première étape.
  \item Effectuer la moyenne des valeurs obtenues à la seconde étape.
  \end{enumerate}
\end{Schunk}

Apportons une toute petite modification à l'algorithme ci-dessus.

\begin{Schunk}
  \begin{enumerate}
  \item Sélectionner les données supérieures à $12$.
  \item Effectuer la moyenne des valeurs obtenues à la première étape.
  \item Soustraire $12$ de la valeur retenue à la seconde étape.
  \end{enumerate}
\end{Schunk}

Mathématiquement, les deux approches sont tout à fait équivalentes.
Cependant, avec la première approche, nous effectuons une soustraction
pour chaque valeur supérieure à $12$ que compte le jeu de données. Avec
la seconde approche, nous n'effectuons qu'une seule soustraction. Si
le jeu de données compte un million d'entrées supérieures à $12$,
c'est un million de soustractions contre une seule. En temps de
calcul, cela ne représente qu'un écart de quelques centièmes de
secondes sur un ordinateur moderne, mais ces fractions de secondes
peuvent finir par faire une différence importante lorsque la taille
des jeux de données augmente ou lorsque l'opération se répète de
nombreuses fois.

Attardons-nous à un second exemple intéressant: l'élévation d'une
valeur $b$ à la puissance $n$. Un premier algorithme, écrit cette fois
en \index{pseudo-langage}\emph{pseudo-langage}, pourrait se lire
ainsi.

\begin{Schunk}
\begin{Verbatim}
Puissance(nombre réel b, entier n)
  Si n = 0, retourner 1.
  Retourner b * Puissance(b, n - 1).
Fin Puissance
\end{Verbatim}
\end{Schunk}

Cet algorithme est \index{algorithme!récursif}\emph{récursif}: la
procédure \code{Puissance} s'appelle elle-même à répétition jusqu'à
l'obtention du résultat désiré. Mathématiquement, l'algorithme revient
à calculer $b^n$ de la manière la plus simple et intuitive qui soit:
\begin{equation*}
  b^n = b (b (b \cdots (b))).
\end{equation*}
Cela peut convenir lorsque $n$ est petit, mais peut-on faire mieux
pour une «grande» valeur de $n$? Imaginez que vous ne disposez que
d'une calculatrice munie des opérations arithmétiques de base et du
carré, et que vous devez élever un nombre à la puissance, disons,
$21$. Comment feriez-vous pour réduire le nombre d'opérations à entrer
au clavier?

\notebox{Cet exemple est moins fantaisiste qu'il n'y parait! Jusqu'au
  milieu des années 1990, les étudiants en actuariat ne disposaient
  que de ce type de calculatrice pour les examens professionnels. Il
  s'agissait d'un modèle de Texas Instruments affectueusement surnommé
  «TI-0».}

Vous avez pensé à un «algorithme»? Votre idée consiste fort
probablement à diviser la puissance par deux autant de fois que
possible et à élever au carré par la suite. Dans notre exemple, cela
donne
\begin{align*}
  b^{21}
  &= b (b^{20}) \\
  & = b (b^{10})^2 \\
  &= b ((b^5)^2)^2 \\
  & = b ((b (b^4))^2)^2 \\
  &= b ((b (b^2)^2)^2)^2.
\end{align*}
Ce calcul se traduit en algorithme comme suit.

\begin{Schunk}
\begin{Verbatim}
Puissance(nombre réel b, entier n)
  Si n = 0, retourner 1.
    Si n est pair, retourner (Puissance(b, n/2))^2.
    Si n est impair, retourner b * Puissance(b, n - 1).
Fin Puissance
\end{Verbatim}
\end{Schunk}

Pour élever un nombre à la puissance $21$, le premier algorithme
requiert $20$ opérations et le second, seulement $6$. Le dénombrement
du nombre d'opérations requis par un algorithme est un aspect
important de leur analyse. Il se fait généralement en notation $O()$
qui signifie «de l'ordre de». Dans le premier algorithme de calcul de
$b^n$, le nombre d'opérations est directement proportionnel à $n$,
donc on dit que l'algorithme est $O(n)$. Dans le second algorithme où
la puissance est divisée par deux à répétition, le nombre d'opérations
est proportionnel au logarithme (en base deux) de $n$, d'où un
algorithme $O(\log n)$. Pour en savoir plus, consultez
\citet[chapitre~1]{Stephens:algorithms:2013}, qui explique à la fois
très clairement et de manière concise les principes de base du
dénombrement des opérations d'un algorithme.

\citet{Kernighan:practice:1999} font remarquer, fort à propos:
\begin{quote}
  Tous les programmes reposent sur des algorithmes et des structures
  de données, mais très peu de programmes exigent d'en concevoir de
  nouveaux.
\end{quote}
Autrement dit, aussi complexe soit-il, un programme repose souvent sur
quelques algorithmes fondamentaux bien connus et bien établis. À ce
titre, les algorithmes de tri et de recherche jouent un rôle
particulièrement important: on estime que 25~\% du temps de calcul des
ordinateurs dans le monde est consacré au tri et à la recherche! Ce
n'est pas pour rien que Donald Knuth consacre un volume entier de son
œuvre monumentale \emph{The Art of Computer Programming} à ce seul
sujet \citep{Knuth:ACP:vol1:1997}.

L'étude des algorithmes classiques de tri et de recherche fera l'objet
du \autoref{chap:algorithmique}.



\section{Algorithmes classiques}
\label{sec:algorithmique:tri}

Pour l'étude des algorithmes classiques, je m'en remets à l'excellent
ouvrage \emph{Essential Algorithms} de
\citet{Stephens:algorithms:2013}. Rédigé dans un style clair et
direct, le manuscrit va juste assez loin pour nos besoins dans les
différentes notions, sans pour autant sacrifier les détails
importants. Bref, un juste équilibre entre «algorithmique pour les
nuls» et un livre plus théorique.

Je vous invite donc à étudier les chapitres suivants de \emph{Essential
Algorithms}:
\begin{itemize}
\item chapitre 6, pages 131--138, 145 (à partir de la section
  \emph{Quicksort})--148 et 156--160;
\item chapitre 7, pages 163--168;
\item chapitre 9, pages 185--193.
\end{itemize}

Quelques remarques pour vous aider dans votre lecture.
\begin{itemize}
\item Partout où vous lirez \emph{array}, pensez tout simplement à un
  vecteur dans R.
\item Tel que mentionné au chapitre~1 de
  \citet{Stephens:algorithms:2013}, l'auteur a pris le parti d'indicer
  les vecteurs comme dans plusieurs langages de programmation, soit de
  $0$ à $n - 1$ pour un vecteur de longueur $n$. Vous devrez adapter
  les algorithmes en conséquence pour la programmation en R où les
  vecteurs sont plutôt indicés de $1$ à $n$.
\item Le livre fait ici et là référence au concept de \emph{library}.
  Une bibliothèque (le nom en français) est une collection de
  fonctions ou de sous-programmes que l'on peut utiliser dans nos
  programmes. Certains langages de programmation sont constitués d'un
  noyau assez restreint auquel on ajoute plusieurs fonctionnalités
  standards (le tri ou la recherche, par exemple) par le biais de
  bibliothèques. C'est en partie le cas de R, mais le système de base
  contient déjà un grand nombre de fonctionnalités.
\end{itemize}


\section{Exercices}
\label{sec:algorithmique:exercices}

\Opensolutionfile{solutions}[solutions-algorithmique]

\begin{Filesave}{solutions}
\section*{Chapitre \ref*{chap:algorithmes}}
\addcontentsline{toc}{section}{Chapitre \protect\ref*{chap:algorithmes}}

\begingroup
%% modifications locales au style des listings
\lstset{%
  frame=single,
  numbers=none,
  basicstyle=\normalfont\ttfamily\NoAutoSpacing}

\end{Filesave}



\begin{Filesave}{solutions}
\endgroup                               % fin de \lstset local
\end{Filesave}

\Closesolutionfile{solutions}

%%% Local Variables:
%%% mode: latex
%%% TeX-engine: xetex
%%% TeX-master: "programmer-avec-r"
%%% coding: utf-8
%%% End:
