\section*{Chapitre \ref{avance}}
\begin{reponse}{6.1}
    Soit \code{Xij} et \code{wij} des matrices, et \code{Xijk}
    et \code{wijk} des tableaux ^^e0 trois dimensions.
    \begin{enumerate}
\item
\begin{Schunk}
\begin{Sinput}
> rowSums(Xij * wij)/rowSums(wij)
\end{Sinput}
\end{Schunk}
\item
\begin{Schunk}
\begin{Sinput}
> colSums(Xij * wij)/colSums(wij)
\end{Sinput}
\end{Schunk}
\item
\begin{Schunk}
\begin{Sinput}
> sum(Xij * wij)/sum(wij)
\end{Sinput}
\end{Schunk}
\item
\begin{Schunk}
\begin{Sinput}
> apply(Xijk * wijk, c(1, 2), sum)/apply(wijk,
+     c(1, 2), sum)
\end{Sinput}
\end{Schunk}
\item
\begin{Schunk}
\begin{Sinput}
> apply(Xijk * wijk, 1, sum)/apply(wijk, 1,
+     sum)
\end{Sinput}
\end{Schunk}
\item
\begin{Schunk}
\begin{Sinput}
> apply(Xijk * wijk, 2, sum)/apply(wijk, 2,
+     sum)
\end{Sinput}
\end{Schunk}
\item
\begin{Schunk}
\begin{Sinput}
> sum(Xijk * wijk)/sum(wijk)
\end{Sinput}
\end{Schunk}
    \end{enumerate}
  
\end{reponse}
\begin{reponse}{6.2}
    \begin{enumerate}
\item
\begin{Schunk}
\begin{Sinput}
> unlist(lapply(0:10, seq, from = 0))
\end{Sinput}
\end{Schunk}
\item
\begin{Schunk}
\begin{Sinput}
> unlist(lapply(1:10, seq, from = 10))
\end{Sinput}
\end{Schunk}
\item
\begin{Schunk}
\begin{Sinput}
> unlist(lapply(10:1, seq, to = 1))
\end{Sinput}
\end{Schunk}
    \end{enumerate}
  
\end{reponse}
\begin{reponse}{6.3}
    \begin{enumerate}
\item
\begin{Schunk}
\begin{Sinput}
> ea <- lapply(seq(100, 300, by = 50), rpareto,
+     alpha = 2, lambda = 5000)
\end{Sinput}
\end{Schunk}
\item
\begin{Schunk}
\begin{Sinput}
> names(ea) <- paste("echantillon", 1:5, sep = "")
\end{Sinput}
\end{Schunk}
\item
\begin{Schunk}
\begin{Sinput}
> sapply(ea, mean)
\end{Sinput}
\end{Schunk}
\item
\begin{Schunk}
\begin{Sinput}
> lapply(ea, function(x) sort(ppareto(x, 2,
+     5000)))
> lapply(lapply(ea, sort), ppareto, alpha = 2,
+     lambda = 5000)
\end{Sinput}
\end{Schunk}
\item
\begin{Schunk}
\begin{Sinput}
> hist(ea$echantillon5)
\end{Sinput}
\end{Schunk}
\item
\begin{Schunk}
\begin{Sinput}
> lapply(ea, "+", 1000)
\end{Sinput}
\end{Schunk}
    \end{enumerate}
  
\end{reponse}
\begin{reponse}{6.4}
    \begin{enumerate}
\item
\begin{Schunk}
\begin{Sinput}
> mean(sapply(x, function(liste) liste$franchise))
\end{Sinput}
\end{Schunk}
Les crochets utilis^^e9s pour l'indi^^e7age constituent en fait un
op^^e9rateur dont le ^^abnom^^bb est \fonction{[[}. On peut donc utiliser cet
op^^e9rateur dans la fonction \code{sapply}:
\begin{Schunk}
\begin{Sinput}
> mean(sapply(x, "[[", "franchise"))
\end{Sinput}
\end{Schunk}
\item
\begin{Schunk}
\begin{Sinput}
> sapply(x, function(x) mean(x$nb.acc))
\end{Sinput}
\end{Schunk}
\item
\begin{Schunk}
\begin{Sinput}
> sum(sapply(x, function(x) sum(x$nb.acc)))
\end{Sinput}
\end{Schunk}
ou
\begin{Schunk}
\begin{Sinput}
> sum(unlist(sapply(x, "[[", "nb.acc")))
\end{Sinput}
\end{Schunk}
\item
\begin{Schunk}
\begin{Sinput}
> mean(unlist(lapply(x, "[[", "montants")))
\end{Sinput}
\end{Schunk}
\item
\begin{Schunk}
\begin{Sinput}
> sum(sapply(x, function(x) sum(x$nb.acc) ==
+     0))
\end{Sinput}
\end{Schunk}
\item
\begin{Schunk}
\begin{Sinput}
> sum(sapply(x, function(x) x$nb.acc[1] ==
+     1))
\end{Sinput}
\end{Schunk}
\item
\begin{Schunk}
\begin{Sinput}
> var(unlist(lapply(x, function(x) sum(x$nb.acc))))
\end{Sinput}
\end{Schunk}
\item
\begin{Schunk}
\begin{Sinput}
> sapply(x, function(x) var(x$nb.acc))
\end{Sinput}
\end{Schunk}
\item
\begin{Schunk}
\begin{Sinput}
> y <- unlist(lapply(x, "[[", "montants"))
> sum(y <= x)/length(y)
\end{Sinput}
\end{Schunk}
La fonction \fonction{ecdf} retourne une fonction permettant
de calculer la fonction de r^^e9partition empirique en tout point:
\begin{Schunk}
\begin{Sinput}
> ecdf(unlist(lapply(x, "[[", "montants")))(x)
\end{Sinput}
\end{Schunk}
\item
\begin{Schunk}
\begin{Sinput}
> y <- unlist(lapply(x, "[[", "montants"))
> colSums(outer(y, x, "<="))/length(y)
\end{Sinput}
\end{Schunk}
La fonction retourn^^e9e par \fonction{ecdf} accepte un vecteur
de points en argument:
\begin{Schunk}
\begin{Sinput}
> ecdf(unlist(lapply(x, "[[", "montants")))(x)
\end{Sinput}
\end{Schunk}
    \end{enumerate}
  
\end{reponse}
