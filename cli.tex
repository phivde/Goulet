%%% Copyright (C) 2018 Vincent Goulet
%%%
%%% Ce fichier fait partie du projet
%%% «Programmer avec R»
%%% https://gitlab.com/vigou3/programmer-avec-r
%%%
%%% Cette création est mise à disposition selon le contrat
%%% Attribution-Partage dans les mêmes conditions 4.0
%%% International de Creative Commons.
%%% https://creativecommons.org/licenses/by-sa/4.0/

\chapter{Ligne de commande}
\label{chap:cli}

\begin{objectifs}
\item Naviguer dans le système de fichiers d'un ordinateur et obtenir
  une liste des fichiers à partir d'une interface en ligne de
  commande.
\item Afficher le contenu d'un fichier à partir de la ligne de
  commande.
\item Utiliser les opérateurs de transfert de données et de
  redirection de la ligne de commande Unix.
\end{objectifs}

Une interface en \index{ligne de commande}ligne de commande
(\emph{command line interface}, CLI), ou tout simplement \emph{ligne
  de commande}, est un mode d'interaction avec un programme
informatique dans lequel l'utilisateur dicte les commandes et reçoit
les réponses de l'ordinateur en mode texte. Le programme qui gère
cette interface est un \emph{interpréteur de commandes}, parfois aussi
appelé (à tort) \emph{terminal} ou \emph{shell}. Lorsque
l'interpréteur est prêt à recevoir une commande, il l'indique par une
\emph{invite de commande} (\emph{command prompt}).

La ligne de commande est la plus ancienne des interfaces avec les
ordinateurs. Si les interfaces graphiques ont à plusieurs égards
grandement facilité l'interaction homme--machine, elles n'ont pas pour
autant fait disparaitre ou rendu obsolète la ligne de commande.
D'abord, celle-ci demeure parfois la seule --- et souvent la plus
rapide --- interface pour réaliser certaines tâches sur un ordinateur
muni d'une interface graphique. Ensuite, certains ordinateurs, comme
les serveurs ou les supercalculateurs, n'offrent souvent aucune
interface graphique.

Dans le cadre du présent ouvrage, le détour que nous effectuons par la
ligne de commande s'explique par le fait que tant les outils d'analyse
et de contrôle de texte \icode{grep}, \icode{sed} et \icode{awk},
étudiés au \autoref{chap:texte}, que l'outil de gestion de versions
Git, qui fait l'objet du \autoref{chap:git}, s'utilisent d'abord et
avant tout à partir de cette interface.


\section{Ligne de commande Windows}
\label{sec:cli:windows}

L'interpréteur de commandes standard sous Windows est
\index{cmd@\code{cmd.exe}}\code{cmd.exe}. On le trouve dans le groupe
de programmes \code{Accessoires} sous le nom \code{Invite de
  commandes}. L'invite de commande par défaut est \verb=C:\>=. À la
gauche du symbole \verb=>= se trouve le chemin d'accès du répertoire
courant. Au lancement de l'interpréteur, le répertoire courant est
donc la racine du système de fichiers.

Les programmeurs qui souhaitent bénéficier de certains outils
\index{Unix}Unix sous Windows peuvent installer les couches
logicielles \index{MinGW}MinGW et \index{MSYS}MSYS --- je recommande à
cet égard la distribution
\link{http://www.msys2.org}{MSYS2}\index{MSYS2}. Le programme
\code{MSYS} fournit un puissant interpréteur de commandes
\index{Bash}Bash \citep{bash} comme en standard sur les systèmes Unix.
Un interpréteur Bash est également fourni avec \index{Git}
\link{https://git-scm.com/downloads}{Git for Windows}: le programme
est d'ailleurs nommé \index{Git~Bash}Git~Bash.

Les fonctionnalités de \code{cmd.exe} étant plutôt limitées, je
recommande fortement d'utiliser MSYS ou Git~Bash comme interpréteur de
commandes sous Windows.

\tipbox{À la ligne de commande \index{Unix}Unix sous Windows, on
  identifie la racine \code{C:\bs} du système de fichier d'un disque
  par \code{/c/}.}


\section{Ligne de commande macOS}
\label{sec:cli:macos}

Sous macOS, l'application de ligne de commande se nomme
\index{Terminal}Terminal. Elle se trouve dans le sous-dossier
Utilitaires du dossier Applications. L'interpréteur de commandes est
\index{Bash}Bash. L'invite de commande par défaut est le symbole
\verb=$=, habituellement précédé du nom de l'ordinateur, du répertoire
courant et du nom de l'utilisateur. Le répertoire courant au lancement
de l'interpréteur est le répertoire personnel \,\verb=~=\, de
l'utilisateur.

\tipbox{Explorez les préférences de l'application
  \index{Terminal}Terminal pour configurer la police de caractères et
  les couleurs afin de rendre la ligne de commande plus agréable à
  utiliser.}


\section{Commandes essentielles}
\label{sec:cli:commandes}

Les programmeurs doivent connaitre les rudiments de la ligne de
commande. Cela dit, comme nous ne pouvons nous étendre sur le sujet,
nous limiterons la présentation aux commandes ci-dessous qui
permettent de se déplacer dans le système de fichiers, d'afficher la
liste des fichiers d'un répertoire et d'afficher le contenu d'un
fichier.

\importantbox{Dans les exemples ci-dessous et pour le reste du
  document, l'invite de commande du système d'exploitation sera
  représentée par le symbole \code{\$} précédé du nom du répertoire
  courant (lorsque cela s'avère pertinent).}

\begin{ttscript}{pwd}
\item[\Icode{cd}] Change le répertoire courant pour
  celui donné en argument. Cet argument peut être un chemin d'accès
  absolu ou relatif (\autoref{sec:informatique:fs:path}). Le nom
  fictif «\verb=..=» identifie le parent du répertoire courant.

  Lorsque utilisée sans argument, la commande affiche le chemin
  d'accès absolu du répertoire courant avec \code{cmd.exe}, ou ramène
  directement l'invite de commande au répertoire personnel de
  l'utilisateur avec Bash.
  \begin{Schunk}
\begin{Verbatim}
~$ cd Desktop/
/Users/vincent/Desktop
~/Desktop$ cd ~/Documents/cours/
~/Documents/cours$ cd
~$
\end{Verbatim}
  \end{Schunk}
\item[\Icode{pwd}] (\index{Bash}Bash seulement) Affiche le chemin
  d'accès absolu du répertoire courant.
  \begin{Schunk}
\begin{Verbatim}
~$ pwd
/Users/vincent
\end{Verbatim}
  \end{Schunk} %$
\item[\Icode{dir}] (\index{cmd@\code{cmd.exe}}\code{cmd.exe}
  seulement) Affiche les fichiers du répertoire donné en argument, ou
  les fichiers du répertoire courant sans aucun argument.
\item[\code{ls}] \Index{ls@\code{ls} (Bash)} (\index{Bash}Bash
  seulement) Affiche les fichiers du répertoire donné en argument, ou
  les fichiers du répertoire courant sans aucun argument.
  \begin{Schunk}
\begin{Verbatim}[commandchars=\\\{\},samepage=true]
~$ ls
Desktop
Documents
Downloads
\meta{...}
\end{Verbatim}
  \end{Schunk} %$
\item[\Icode{cat}] (\index{Bash}Bash seulement) Affiche le contenu du
  fichier donné en argument. Utile uniquement pour les fichiers en
  format texte.
  \begin{Schunk}
\begin{Verbatim}
~$ cat hello.txt
Hello World!
\end{Verbatim}
  \end{Schunk} %$
\end{ttscript}

\tipbox{Les systèmes d'exploitation affichent certains répertoires
  (\code{Bureau} ou \code{Musique}, par exemple) dans la langue de
  l'interface plutôt que sous leur véritable nom, qui est en anglais.
  La commande \code{ls} affiche toujours le véritable nom des
  répertoires.}


\section{Transfert de données et redirection}
\label{sec:cli:flux}

Les programmes Unix en ligne de commande reçoivent leurs données d'une
\Index{entree standard@entrée standard}\Index{Unix!entree
  standard@entrée standard}\emph{entrée standard} (\emph{standard
  input}, souvent abrégé en \emph{stdin}) et renvoient leurs résultats
vers la \Index{sortie standard}\Index{Unix!sortie
  standard}\emph{sortie standard} (\emph{standard output},
\emph{stdout}). Pour simplifier, l'entrée standard d'un terminal
serait le clavier et la sortie standard, l'écran.

L'opérateur de transfert de données %
\index{{"|}@\code{\textbar} (Bash)|bfhyperpage}%
\index{tuyau|see{\code{\textbar} (Bash)}} %
«\code{\textbar}» (nommé «tuyau», de l'anglais \emph{pipe}) permet de
transférer la sortie d'un programme directement à l'entrée d'un autre
programme. Autrement dit, plutôt que d'afficher ses résultats à
l'écran, le premier programme les passe en entrée au second programme.
L'outil \icode{cat} est fréquemment utilisé comme point de départ.
\begin{Schunk}
\begin{Verbatim}
~$ cat hello.txt | tr ! ?
Hello World?
~$ pwd
/Users/vincent
~$ pwd | cut -d / -f 3
vincent
\end{Verbatim}
\end{Schunk} %$

\tipbox{La commande \Icode{tr} transforme le premier caractère en
  argument par le second dans le texte en entrée standard. La commande
  \Icode{cut}, quant à elle, scinde le texte en entrée standard en
  champs délimités par le caractère suivant l'option \code{-d} et
  retourne le champ \meta{n} tel que spécifié par l'option \code{-f
    \meta{n}}.}

L'opérateur de redirection %
\index{>@\code{>} (Bash)|bfhyperpage}«\code{>}» permet, lui, de
rediriger la sortie d'un programme vers un fichier afin de sauvegarder
les résultats. Si le fichier existe déjà, son contenu est écrasé par
le nouveau contenu. Pour ajouter le nouveau contenu à la fin du
fichier, il faut utiliser la variante double %
\index{>>@\code{>>} (Bash)|bfhyperpage}«\code{>>}» de l'opérateur de
redirection.
\begin{Schunk}
\begin{Verbatim}
~$ echo 'Hello World!' > hello.txt
~$ cat hello.txt
Hello World!
~$ echo 'Salut la compagnie!' >> hello.txt
~$ cat hello.txt
Hello World!
Salut la compagnie!
\end{Verbatim}
\end{Schunk}

Enfin, l'opérateur de redirection %
\Index{>@\code{<} (Bash)}«\code{<}» permet, à l'inverse du précédent,
de déverser le contenu d'un fichier dans l'entrée standard d'un
programme. La variante double existe également.
\begin{Schunk}
\begin{Verbatim}
$ tr ! ? < hello.txt
Hello World?
Salut la compagnie?
\end{Verbatim}
\end{Schunk} %$

La documentation citée en référence au \autoref{chap:texte} utilise
abondamment les opérateurs ci-dessus.


\section{Exercices}
\label{sec:cli:exercices}

\Opensolutionfile{solutions}[solutions-cli]

\begin{Filesave}{solutions}
\section*{Chapitre \ref*{chap:cli}}
\addcontentsline{toc}{section}{Chapitre \protect\ref*{chap:cli}}

\end{Filesave}

\begin{exercice}
  Démarrer un interpréteur de commandes Bash. Si le système
  d'exploitation de votre ordinateur est Windows, vous pouvez utiliser
  la ligne de commande \index{Git~Bash}Git~Bash de
  \link{https://git-scm.com/downloads}{\emph{Git for Windows}} ou la
  ligne de commande MSYS de
  \index{MSYS2}\link{http://www.msys2.org}{MSYS2}.
  \begin{enumerate}
  \item Afficher le répertoire courant. Vérifier qu'il correspond à
    celui mentionné dans l'invite de commande. Localiser ce répertoire
    dans l'interface graphique de votre système d'exploitation (Finder
    sous macOS, Windows~Explorer sous Windows).
  \item Afficher la liste des fichiers du répertoire courant.
  \item Choisir un sous-répertoire du répertoire courant que vous
    savez contenir lui-même un sous-répertoire (il pourrait par
    exemple s'agir de \code{Documents/Cours}). Afficher, sans d'abord
    s'y déplacer, la liste des fichiers du premier sous-répertoire
    (\code{Documents}), puis celle du second (\code{Cours}).
  \item Faire du sous-répertoire utilisé précédemment le répertoire
    courant.
  \item Afficher une liste détaillée des fichiers du nouveau
    répertoire courant avec la commande \verb=ls -l=.
  \item Descendre d'un autre niveau dans l'arbre des fichiers.
  \item Revenir au répertoire personnel avec une seule commande.
  \item Afficher la liste de tous les fichiers du répertoire
    personnel, y compris les fichiers cachés, avec la commande
    \verb=ls -a=.
  \item Afficher la liste détaillée de tous fichiers du répertoire
    personnel avec la commande \verb=ls -la=.
  \item Afficher la liste des répertoires dans lesquels le système
    d'exploitation recherche des applications avec la commande
    \verb=echo $PATH=.
  \end{enumerate}
  \begin{sol}
    Vous trouverez ci-dessous une transcription (partielle) de la
    séance de travail à effectuer depuis une ligne de commande Bash.
    \begin{Schunk}
\begin{Verbatim}[commandchars=\\\{\}]
\textbf{~$} pwd
/Users/vincent
\textbf{~$} ls
Desktop           Music        emacs-modified-extensions
Documents         Pictures     emacs-modified-macos
Downloads         Public       emacs-modified-windows
\meta{...}
\textbf{~$} ls Documents/
Spoon-Knife
actexam
actuar
actuarialangle
actuarialsymbol
\meta{...}
\textbf{~$} ls Documents/cours/
ACT-2002        ACT-2005        ACT-2008        IFT-1902
\textbf{~$} cd Documents/
/Users/vincent/Documents
\textbf{~/Documents$} ls -l
drwxr-xr-x   6 vincent staff 204 29 nov 14:40 Spoon-Knife
drwxr-xr-x   6 vincent staff 204 21 nov  2016 actexam
drwxr-xr-x  10 vincent staff 340  1 sep 09:56 actuar
\meta{...}
\textbf{~/Documents$} cd cours/
/Users/vincent/Documents/cours
\textbf{~/Documents/cours$} cd
\textbf{~$} ls -a
.                     .bashrc           .profile
..                    .cache            .rstudio-desktop
.CFUserTextEncoding   .cups             .subversion
\meta{...}
\textbf{~$} ls -la
total 584
drwxr-xr-x@ 71 vincent staff  2414  4 déc 21:03 .
drwxr-xr-x   5 root    admin   170 19 oct 04:13 ..
drwxr-xr-x   4 vincent staff   136 10 mai  2017 .R
-rw-r--r--   1 vincent staff    63  5 jul  2013 .Renviron
-rw-r--r--   1 vincent staff 23185  9 nov 09:13 .Rhistory
-rw-r--r--   1 vincent staff    71  4 jan  2016 .Rprofile
\meta{...}
\textbf{~$} echo PATH
/opt/local/bin:/opt/local/sbin:~/bin:/usr/local/bin:
/usr/bin:/bin:/usr/sbin:/sbin:/Library/TeX/texbin
\end{Verbatim}
    \end{Schunk} %$
  \end{sol}
\end{exercice}

\begin{exercice}
  Créer un fichier nommé \code{foobar} depuis la ligne de commande
  avec la commande:
  \begin{Schunk}
\begin{Verbatim}
~$ echo 'aaa.bbb.ccc' > foobar
\end{Verbatim}
  \end{Schunk} %$
  Déterminer ensuite le résultat des commandes suivantes.
  \begin{enumerate}
  \item \code{cat foobar}
  \item \code{cat foobar | cut -d . -f 1}
  \item \code{cat foobar | tr a z}
  \item \code{tr a z < foobar}
  \end{enumerate}
  \begin{sol}
    \begin{enumerate}
    \item La commande affiche le contenu du fichier \code{foobar}
      \begin{Schunk}
\begin{Verbatim}
aaa.bbb.ccc
\end{Verbatim}
      \end{Schunk}
    \item La commande transfère le contenu du fichier \code{foobar}
      dans l'outil \icode{cut}, qui le découpe en champs délimités par
      le point et retourne le premier champ.
      \begin{Schunk}
\begin{Verbatim}
aaa
\end{Verbatim}
      \end{Schunk}
    \item La commande déverse le contenu du fichier \code{foobar} dans
      l'outil \icode{tr} qui convertit tous les \code{a} en \code{z}.
      \begin{Schunk}
\begin{Verbatim}
zzz.bbb.ccc
\end{Verbatim}
      \end{Schunk}
    \item Même commande qu'en c), mais en utilisant la redirection
      vers l'entrée standard plutôt que le transfert de données.
      \begin{Schunk}
\begin{Verbatim}
zzz.bbb.ccc
\end{Verbatim}
      \end{Schunk}
    \end{enumerate}
  \end{sol}
\end{exercice}

\Closesolutionfile{solutions}

%%% Local Variables:
%%% mode: latex
%%% TeX-engine: xetex
%%% TeX-master: "programmer-avec-r"
%%% coding: utf-8
%%% End:
