%%% Copyright (C) 2018 Vincent Goulet
%%%
%%% Ce fichier fait partie du projet
%%% «Programmer avec R»
%%% https://gitlab.com/vigou3/programmer-avec-r
%%%
%%% Cette création est mise à disposition selon le contrat
%%% Attribution-Partage dans les mêmes conditions 4.0
%%% International de Creative Commons.
%%% https://creativecommons.org/licenses/by-sa/4.0/

\chapter{Bibliothèques et paquetages}
\label{chap:paquetages}

\def\scriptfilename{\currfilebase.R}

\begin{objectifs}
\item Distinguer les concepts de bibliothèque et de paquetage dans R.
\item Utiliser les fonctionnalités d'un paquetage dans une session de
  travail R.
\item Installer de nouveaux paquetages R depuis le site
  \emph{Comprehensive R Archive Network} (CRAN).
\end{objectifs}

L'un des aspects de R ayant sans aucun doute le plus contribué à son
succès est la possibilité --- et la facilité --- d'ajouter des
extensions au système de base par le biais de paquetages. Toute
personne utilisant R sera un jour appelée à utiliser des paquetages.
Ce chapitre explique comment ajouter une bibliothèque personnelle au
système et comment la garnir de paquetages téléchargés depuis le site
\emph{Comprehensive R Archive Network} (CRAN).

Dans la terminologie de R, un \index{paquetage}\emph{paquetage}
(\emph{package}\footnote{%
  L'équipe de traduction française de R a choisi de conserver le terme
  anglais. Je ne les suis pas dans cette voie.}). %
est un ensemble cohérent de fonctions, de jeux de données et de
documentation. Les paquetages sont regroupés dans une
\emph{bibliothèque} (\emph{library}).

\warningbox{Insistons sur la terminologie puisque la confusion règne
  souvent hors de la documentation officielle de R: un paquetage est
  un ensemble de fonctions, alors qu'une bibliothèque est un ensemble
  de paquetages.}


\section{Système de base}
\label{sec:paquetages:base}

La bibliothèque de base de R contient une trentaine de paquetages dont
certains sont chargés par défaut au démarrage. La fonction
\icode{search} fournit la liste des paquetages chargés dans la session
de travail, alors que \icode{library} affiche le contenu de la
bibliothèque de paquetages.
\begin{Schunk}
\begin{Sinput}
> search()
\end{Sinput}
\begin{Soutput}
[1] ".GlobalEnv"        "package:stats"
[3] "package:graphics"  "package:grDevices"
[5] "package:utils"     "package:datasets"
[7] "package:methods"   "Autoloads"
[9] "package:base"
\end{Soutput}
\begin{Sinput}
> library()
\end{Sinput}
\begin{Verbatim}[commandchars=\\\{\}]
Packages dans la bibliothèque
‘/Library/Frameworks/R.framework/\meta{...}/library’ :

base           The R Base Package
boot           Bootstrap Functions
               (Originally by Angelo Canty for S)
\meta{...}
\end{Verbatim}
\end{Schunk}


\section{Utilisation d'un paquetage}
\label{sec:paquetages:utilisation}

Pour rendre disponibles les fonctionnalités d'un paquetage, peu
importe sa provenance, il faut le charger dans la session de travail.
C'est là le principal rôle de la fonction \Icode{library}.
\begin{Schunk}
\begin{Verbatim}[commandchars=\\\{\}]
library("\meta{paquetage}")
\end{Verbatim}
\end{Schunk}


\section{Création d'une bibliothèque personnelle}
\label{sec:paquetages:library}

Nous verrons à la \autoref{sec:paquetages:install} comment ajouter des
paquetages au système de base. Je recommande fortement de prévoir une
bibliothèque personnelle pour contenir ces nouveaux paquetages. Tout
d'abord, cela permet d'éviter les éventuels problèmes d'accès en
écriture dans la bibliothèque principale, surtout sur des systèmes
partagés. Ensuite, les paquetages placés dans une bibliothèque
personnelle résistent mieux au processus de mise à jour de R que ceux
ajoutés à la bibliothèque principale.

Avant toute chose, il faut créer un répertoire dans son système de
fichiers qui servira de bibliothèque personnelle. Dans la suite, nous
utiliserons \verb=~/R/library=, où \verb=~= représente le dossier de
départ (\autoref{sec:informatique:os:unix}).

\osxbox{Sous macOS, le dossier tout indiqué pour la bibliothèque
  personnelle est \verb=~/Library/R/library=. Toutefois, le dossier
  \verb=~/Library= est caché dans le Finder. Pour y accéder, cliquez
  sur le menu \code{Aller} en appuyant simultanément sur la touche
  Option (\optkey), puis sélectionnez \code{Bibliothèque}.}

Au démarrage, R lit le fichier de configuration \verb=~/.Renviron=,
s'il existe. C'est dans ce fichier que l'on indique à R l'existence
d'une bibliothèque personnelle de même que l'endroit où elle se trouve
dans le système de fichiers.

À l'aide d'un éditeur de texte, ouvrir ou créer\footnote{Pour créer le
  fichier dans RStudio, sélectionner \code{File|New File|Text
    File}.} %
le fichier \verb=~/.Renviron= et y ajouter la ligne ci-dessous. Au
besoin, remplacer le chemin \verb=~/R/library= par celui du répertoire
créé précédemment. Utiliser la barre oblique avant (\code{/}) pour
séparer les répertoires dans le chemin d'accès.
\begin{Schunk}
\begin{verbatim}
R_LIBS_USER="~/R/library"
\end{verbatim}
\end{Schunk}

\osxbox{Pour afficher les fichiers débutant par un point dans la boite
  de dialogue d'ouverture de fichier d'une application comme RStudio,
  appuyez sur \shiftkey\cmdkey\code{.} (\shiftkey\cmdkey~«point»).}

Une fois la configuration terminée et après un redémarrage de R, le
chemin vers la bibliothèque personnelle devrait apparaitre dans les
résultats de la commande
\Index{libPaths@\code{.libPaths}}\code{.libPaths}.
\begin{Schunk}
\begin{Sinput}
> .libPaths()
\end{Sinput}
\begin{Verbatim}[commandchars=\\\{\}]
[1] "/Users/vincent/Library/R/library"
[2] "/Library/Frameworks/R.framework/\meta{...}/library"
\end{Verbatim}
\end{Schunk}

La bibliothèque personnelle a toujours préséance sur la bibliothèque
principale.

\tipbox{Consultez la rubriques d'aide de \icode{Startup} pour les
  détails sur la syntaxe et l'emplacement des fichiers de
  configuration et celles de \icode{library} et de
  \index{libPaths@\code{.libPaths}}\code{.libPaths} pour la gestion
  des bibliothèques.}


\section{Installation de paquetages additionnels}
\label{sec:paquetages:install}

Dès les débuts de R, les développeurs et utilisateurs ont mis sur pied
le dépôt central de paquetages
\link{http://cran.r-project.org}{\emph{Comprehensive R Archive
    Network}} (CRAN). Ce site compte aujourd'hui des milliers de
paquetages et le nombre ne cesse de croitre.

L'installation d'un paquetage distribué via CRAN est très simple avec
la fonction \Icode{install.packages}. Celle-ci télécharge le paquetage
spécifié en argument depuis CRAN et l'installe, par défaut, dans la
première bibliothèque de
\index{libPaths@\code{.libPaths}}\code{.libPaths}.
\begin{Schunk}
\begin{Verbatim}[commandchars=\\\{\}]
install.packages("\meta{paquetage}")
\end{Verbatim}
\end{Schunk}

Pour utiliser le paquetage, il faut le charger en mémoire avec
\icode{library}, tel qu'expliqué à la
\autoref{sec:paquetages:utilisation}. R recherche le paquetage
spécifié dans les bibliothèques de
\index{libPaths@\code{.libPaths}}\code{.libPaths}.

\gotorbox{Étudiez le fichier de script \code{\scriptfilename}
  reproduit à la \autoref{sec:paquetages:exemples}.}


\section{Exemples}
\label{sec:paquetages:exemples}

\scriptfile{\scriptfilename}
\lstinputlisting[firstline=\scriptfirstline]{\scriptfilename}


\section{Exercices}
\label{sec:paquetages:exercices}

\begin{exercice}[nosol]
  Installer un paquetage disponible dans CRAN dans une bibliothèque
  personnelle sur votre poste de travail.
\end{exercice}

%%% Local Variables:
%%% mode: noweb
%%% TeX-engine: xetex
%%% TeX-master: "programmer-avec-r"
%%% coding: utf-8
%%% End:
