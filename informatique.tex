%%% Copyright (C) 2017 Vincent Goulet
%%%
%%% Ce fichier fait partie du projet «Programmer avec R»
%%% http://github.com/vigou3/programmer-avec-r
%%%
%%% Cette création est mise à disposition selon le contrat
%%% Attribution-Partage dans les mêmes conditions 4.0
%%% International de Creative Commons.
%%% http://creativecommons.org/licenses/by-sa/4.0/

\chapter{Éléments d'informatique pour programmeurs}
\label{chap:informatique}

Intro du chapitre

\section{Bref historique des langages de programmation}
\label{sec:informatique:historique}

Ada Lovelace (1815--1852) est généralement reconnue comme la première
auteure d'un algorithme et de ce que l'on appelle aujourd'hui un
programme informatique. C'est en son honneur qu'été nommé le langage
Ada conçu en réponse à un cahier de charges du département de la
Défense des États-Unis au début des années 1980.

Franchissons d'un bond les premiers jours de l'informatique et de la
programmation\footnote{%
  Les lecteurs intéressés à en apprendre davantage sur le sujet
  pourront débuter par les entrées de Wikipedia
  (\link{https://fr.wikipedia.org/wiki/Histoire_des_langages_de_programmation}{français},
  \link{https://en.wikipedia.org/wiki/History_of_programming_languages}{anglais}).} %
pour arriver aux ordinateurs électriques modernes, dans les années
1940. On programme alors généralement ceux-ci en \emph{assembleur}, un
langage de très bas niveau facilement interprétable par la machine,
mais difficile à lire par des humains; voir la
\autoref{fig:informatique:assembleur}.

\begin{figure}
  \centering
  \includegraphics[trim=0 475 0 0, clip=true]{Motorola_6800_Assembly_Language.png}
  \caption[Programme assembleur pour un microprocesseur 8~bits
  Motorola 6800.]{Extrait d'un programme en assembleur pour un
    microprocesseur 8~bits Motorola 6800. Source:
    \link{https://commons.wikimedia.org/wiki/File\%3AMotorola_6800_Assembly_Language.png}{Wikimedia Commons}}
  \label{fig:informatique:assembleur}
\end{figure}

Les premiers langages créés pour transmettre des instructions à un
ordinateur apparaissent dans les années 1950. Ils sont à l'origine
intimement liés à l'architecture d'un ordinateur: à chaque type
d'ordinateur son langage de programmation. Certaines contraintes ---
ou exigences --- des langages de l'époque proviennent aussi du support
physique alors utilisé pour stocker les programmes: les cartes
perforées.

En 1954, l'ingénieur de IBM John Backus publie les spécifications du
langage FORTRAN (pour \emph{FORmula TRANslating System}). Le premier
compilateur voit le jour deux ans plus tard. Fortran (avec des
minuscules à partir de 1977) deviendra rapidement le langage standard
dans le calcul scientifique. Plus d'un demi-siècle plus tard, son
empreinte demeure importante, notamment grâce aux bibliothèques
d'algèbre linéaire BLAS\footnote{%
  \emph{Basic Linear Algebra Subprograms};
  \link{http://www.netlib.org/blas}{}.} %
et LAPACK\footnote{%
  \emph{Linear Algebra PACKage};
  \link{http://www.netlib.org/lapack}{}.} %
auxquelles ont recours la plupart des progiciels scientifiques, dont
R.

\begin{figure}[t]
  \notebox{Dans le très recommandable film Les figures de l'ombre
    (\emph{Hidden Figures}, 2016), une des héroïnes entreprend de
    s'attaquer à la programmation des nouveaux ordinateurs de la NASA.
    On peut alors voir qu'elle apprend le Fortran.}
\end{figure}

Autre ancien langage toujours en usage aujourd'hui dans les
applications de gestion, COBOL (\emph{Common Business Oriented
  Language}) a été créé en 1959 par un comité formé pour proposer un
langage commun pour l'administration américaine. La légende urbaine
veut que les programmeurs COBOL soient aujourd'hui fort bien rémunérés
du fait de leur rareté et de l'importance opérationnelle des
applications qu'ils doivent maintenir.

Plus vieux langages encore en activité: COBOL (1959), FORTRAN (1954),
LISP (1958)

Lisp langage beaucoup utilisé en intelligence artificielle;
particulièrement élégant, sauf peut-être pour la syntaxe. Citation:
\emph{}

"Lisp is worth learning for the profound enlightenment experience you will have when you finally get it; that experience will make you a better programmer for the rest of your days, even if you never actually use Lisp itself a lot."
- Eric Raymond, "How to Become a Hacker"
http://www.paulgraham.com/quotes.html

Greenspun's tenth rule of programming is an aphorism in computer programming and especially programming language circles that states:[1][2]

Any sufficiently complicated C or Fortran program contains an ad-hoc,
informally-specified, bug-ridden, slow implementation of half of
Common Lisp.
Version courte: Ceux qui ne connaissent pas le Lisp sont condamnés à
le réinventer
https://en.wikipedia.org/wiki/Greenspun%27s_tenth_rule

\begin{figure}[t]
  \setlength{\FrameRule}{1pt}
  \begin{emphbox}{\mdseries Notations infixée, préfixée et suffixée}
    Les notations infixée (\emph{infix}), préfixée (\emph{prefix}) et
    suffixée (\emph{postfix}) sont trois manières différentes, mais
    équivalentes, d'écrire des expressions en mathématiques ou en
    programmation.

    Par exemple, l'opération d'addition de deux opérandes $x$ et $y$
    s'écrit en notation infixée
\begin{lstlisting}[backgroundcolor=\color{codebg}]
x + y
\end{lstlisting}
    En notation préfixée, aussi appelée notation polonaise,
    l'opérateur est placé \emph{avant} les opérandes:
\begin{lstlisting}[backgroundcolor=\color{codebg}]
+ x y
\end{lstlisting}
    On l'aura compris, en notation suffixée, ou notation polonaise
    inversée, l'opérateur apparait \emph{après} les opérandes:
\begin{lstlisting}[backgroundcolor=\color{codebg}]
x y +
\end{lstlisting}
    Nous sommes davantage habitués à lire la notation infixée, quoique
    la notation préfixée nous soit familière pour les opérateurs à un
    seul opérande (comme la négation) ou pour les appels de fonctions.
    La notation suffixée n'a jamais recours aux parenthèses.

    La compagnie HP commercialise des calculatrices scientifiques
    utilisant la notation suffixée (libellées RPN pour \emph{Reverse
      Polish Notation}) depuis 1972.
  \end{emphbox}
\end{figure}

Infix [infixée], Postfix [suffixée] and Prefix [préfixée] notation (encadré?)
\url{http://www.cs.man.ac.uk/~pjj/cs212/fix.html}


Algol Fin des années 1960 publication, by a committee of American and
European computer scientists, "a new language for algorithms"; the
ALGOL 60 Report (the "ALGOrithmic Language"). Velléité de
standardisation des langages de programmation; évidemment un échec;
néanmoins plusieurs innovations importantes qui feront en sorte qu'un
grand nombre des langages qui verront le jour par la suite seront
considérés comme des descendants d'Algol.

\begin{figure}[t]
  \centering
  \begin{minipage}{0.9\linewidth}
    \setkeys{Gin}{width=\textwidth}
    \includegraphics{standards} \\
    \footnotesize\sffamily%
    Tiré de \href{http://xkcd.com/927/}{XKCD.com}
  \end{minipage}
\end{figure}

APL (1962) longtemps un langage très prisé par les actuaires; probablement
encore du code APL dans certaines compagnies d'assurance; inspiration
pour S et R; continue sa vie aujourd'hui surtout sous J.



C Kernighan et Ritchie, Bell Labs -> John Chambers et S; Unix;
toujours beaucoup utilisé pour le calcul numérique intensif; évolution
vers C++

Python (1991)
\url{https://en.wikipedia.org/wiki/Python_(programming_language)}







\section{Syntaxe et sémantique}
\label{sec:informatique:syntaxe}

\section{Paradigmes de programmation}
\label{sec:informatique:paradigmes}

\section{Systèmes d'exploitation}
\label{sec:informatique:os}



%%% Local Variables:
%%% mode: latex
%%% TeX-engine: xetex
%%% TeX-master: "programmer-avec-r"
%%% End:
