%%% Copyright (C) 2018 Vincent Goulet
%%%
%%% Ce fichier fait partie du projet
%%% «Programmer avec R»
%%% https://gitlab.com/vigou3/programmer-avec-r
%%%
%%% Cette création est mise à disposition selon le contrat
%%% Attribution-Partage dans les mêmes conditions 4.0
%%% International de Creative Commons.
%%% https://creativecommons.org/licenses/by-sa/4.0/

\chapter{Éléments d'informatique pour programmeurs}
\label{chap:informatique}

\begin{objectifs}
\item Nommer et ordonner les grands jalons de l'histoire des langages
  de programmation.
\item Identifier les principaux paradigmes de programmation.
\item Distinguer les concepts de syntaxe et de sémantique d'un langage
  de programmation.
\item Distinguer les concepts de langage compilé et de langage
  interprété.
\item Comparer les forces et faiblesses de divers langages de
  programmation en fonction de divers critères.
\item Comprendre la structure du système de fichiers d'un ordinateur
  et identifier le chemin d'accès vers une ressource dans celui-ci.
\item Naviguer dans le système de fichiers d'un ordinateur et obtenir
  une liste des fichiers à partir d'une interface en ligne de
  commande.
\end{objectifs}


- Programmer c'est avant tout résoudre des problèmes. Plutôt que de le
faire avec des mathématiques, on demande à une machine de le faire
pour nous.
- Les humains ont inventé toutes sortes de langages. Certains
parviennent à mieux exprimer certaines réalités que d'autres. Idem
pour les langages de programmation, qui servent à donner des
instructions à une machine qui ne comprend que deux instructions: vrai
et faux, ouvert ou fermé, 0 ou 1.
- Introduire la notion d'algorithme; parallèle avec les recettes
culinaires.



\section{Bref historique des langages de programmation}
\label{sec:informatique:historique}

Ada Lovelace (1815--1852) est généralement reconnue comme la première
auteure d'un algorithme et de ce que l'on appelle aujourd'hui un
programme informatique. C'est en son honneur qu'a été nommé le langage
Ada conçu en réponse à un cahier de charges du département de la
Défense des États-Unis au début des années 1980.

Franchissons d'un bond les premiers jours de l'informatique et de la
programmation pour arriver aux ordinateurs électriques modernes, dans
les années 1940. On programme alors généralement ceux-ci en
\index{assembleur}assembleur, un langage de très bas niveau facilement
interprétable par la machine, mais difficile à lire par des humains,
comme l'extrait de programme de la
\autoref{fig:informatique:assembleur} le démontre bien.

\begin{figure}
  \centering
  \includegraphics[trim=0 475 0 0, clip=true]{images/Motorola_6800_Assembly_Language.png}
  \caption[Programme en assembleur pour un microprocesseur 8~bits
  Motorola 6800.]{Extrait d'un programme en assembleur pour un
    microprocesseur 8~bits Motorola 6800. {\small Source:
      \link{https://commons.wikimedia.org/wiki/File\%3AMotorola_6800_Assembly_Language.png}{Wikimedia
        Commons}}}
  \label{fig:informatique:assembleur}
\end{figure}

Les premiers langages créés pour transmettre des instructions à un
ordinateur apparaissent dans les années 1950. Ils sont à l'origine
intimement liés à l'architecture d'un ordinateur: à chaque type
d'ordinateur son langage de programmation. Certaines contraintes ---
ou exigences --- des langages de l'époque proviennent aussi du support
physique alors utilisé pour stocker les programmes: les cartes
perforées.

\subsection{Fortran}
\label{sec:informatique:historique:fortran}

En 1954, l'ingénieur de IBM John Backus publie les spécifications du
langage \index{Fortran}FORTRAN (\emph{FORmula TRANslating System}). Le
premier compilateur voit le jour deux ans plus tard. Fortran (avec des
minuscules à partir de 1977) deviendra rapidement le langage standard
dans le calcul scientifique.

Plus d'un demi-siècle plus tard, l'empreinte de Fortran demeure
importante, notamment grâce aux bibliothèques d'algèbre linéaire
BLAS\footnote{%
  \emph{Basic Linear Algebra Subprograms};
  \link{http://www.netlib.org/blas}{}.} %
et LAPACK\footnote{%
  \emph{Linear Algebra PACKage};
  \link{http://www.netlib.org/lapack}{}.} %
auxquelles ont recours la plupart des progiciels scientifiques, dont
R. Le langage est toujours utilisé en calcul haute performance et pour
mesurer le rendement des superordinateurs.

\notebox{Dans le très recommandable film «Les figures de l'ombre»
  (\emph{Hidden Figures}, 2016), une des héroïnes entreprend de
  s'attaquer à la programmation des nouveaux ordinateurs de la NASA.
  On peut alors voir qu'elle apprend le Fortran.}

\subsection{Lisp}
\label{sec:informatique:historique:lisp}

Le deuxième langage le plus ancien toujours largement diffusé est
\index{Lisp}Lisp (\emph{LISt Processing}). Créé par John McCarthy en
1958 en tant que modèle pratique pour représenter des programmes, le
Lisp est devenu le langage de choix pour la recherche et les
applications en intelligence artificielle. Le terme Lisp désigne
aujourd'hui une famille de langages comprenant de nombreux dialectes,
dont Common Lisp, Scheme et Emacs Lisp.

Le Lisp se distingue en outre par une syntaxe simple en
\index{notation!préfixée}notation préfixée (voir encadré), le support
pour la programmation fonctionnelle
(\autoref{sec:informatique:paradigmes}) et la faculté de manipuler le
code source en tant que structure de données. Autre trait distinctif:
la syntaxe du Lisp fait un usage immodéré des parenthèses.

Le Lisp est entouré d'une aura de beauté et d'élégance dont peu
d'autres langages peuvent se targuer. Citons Eric Raymond dans
\link{http://www.catb.org/esr/faqs/hacker-howto.html}{\emph{How to
    Become a Hacker}}:
\begin{quote}
  Il faut apprendre le Lisp pour l'extraordinaire expérience d'éveil
  [\emph{enlightenment experience}] que procure le fait de finalement
  le comprendre; cette expérience fera de vous un meilleur programmeur
  pour toujours, même si vous n'avez plus vraiment à utiliser le Lisp.
\end{quote}

Pour illustrer encore davantage la place toute particulière qu'occupe
le Lisp en programmation, mentionnons également l'aphorisme selon
lequel «ceux qui ne connaissent pas le Lisp sont condamnés à le
réinventer», d'une certaine façon la version courte de la célèbre
\link{https://en.wikipedia.org/wiki/Greenspun\%27s_tenth_rule}{\emph{Greenspun's
    tenth rule of programming}}: %
\begin{quote}
  Tout programme suffisamment complexe en C ou en Fortran comporte une
  mise en œuvre mal spécifiée, pleine de bogues et lente de la moitié
  de Common Lisp.
\end{quote}
(Fait amusant à noter: il n'y a pas d'autres lois que la dixième, de
l'\link{http://philip.greenspun.com/bboard/q-and-a-fetch-msg?msg_id=000tgU}{aveu
  même de l'auteur}.)

\begin{figure}[t]
  \label{fig:informatique:notations}
  \setlength{\FrameRule}{1pt}
  \lstset{backgroundcolor=\color{codebg},
    frame=lr, rulecolor=\color{codebg},
    xleftmargin=3.4pt, xrightmargin=3.4pt}
  \begin{emphbox}{\mdseries Notations infixée, préfixée et suffixée}
    Les notations \index{notation!infixée}infixée (\emph{infix}),
    \index{notation!préfixée}préfixée (\emph{prefix}) et
    \index{notation!suffixée}suffixée (\emph{postfix}) sont trois
    manières différentes, mais équivalentes, d'écrire des expressions
    en mathématiques ou en programmation.

    Par exemple, l'opération d'addition de deux opérandes $x$ et $y$
    s'écrit en notation infixée
\begin{lstlisting}
x + y
\end{lstlisting}
    En notation préfixée, aussi appelée notation polonaise,
    l'opérateur est placé avant les opérandes:
\begin{lstlisting}
+ x y
\end{lstlisting}
    On l'aura compris, en notation suffixée, ou notation polonaise
    inversée, l'opérateur apparait après les opérandes:
\begin{lstlisting}
x y +
\end{lstlisting}
    Nous sommes davantage habitués à lire la notation infixée, quoique
    la notation préfixée nous soit familière pour les opérateurs à un
    seul opérande (comme la négation) ou pour les appels de fonctions.
    La notation suffixée n'a jamais recours aux parenthèses.

    La compagnie HP commercialise de très prisées calculatrices
    scientifiques utilisant la notation suffixée (libellées RPN pour
    \emph{Reverse Polish Notation}) depuis 1972.
  \end{emphbox}
\end{figure}

\subsection{COBOL}
\label{sec:informatique:historique:cobol}

Le troisième langage développé dans les années 1950 et toujours en
usage de nos jours est \index{COBOL}COBOL (\emph{COmmon Business
  Oriented Language}). Ce langage spécialisé dans les applications de
gestion a été créé en 1959 par un comité formé pour proposer un
langage commun pour l'administration américaine.

Le COBOL reste très utilisé dans de grandes entreprises, notamment
dans les institutions financières. La légende urbaine veut d'ailleurs
que les programmeurs COBOL soient comparativement très bien rémunérés
aujourd'hui sous l'effet combiné de leur rareté et de l'importance
opérationnelle des applications qu'ils doivent maintenir.

\subsection{Algol}
\label{sec:informatique:historique:algol}

Dès la fin de la décennie 1950, un comité de scientifiques se réunit à
Zurich pour concevoir ce que l'on voudrait voir devenir le langage de
programmation standard. De ces rencontres naitra \index{Algol}Algol
(\emph{ALGorithmic Oriented Language}) en 1958. Comme la plupart des
tentatives de définition d'un standard, c'est un échec: le langage est
populaire dans les milieux académiques, mais restera peu utilisé dans
les applications commerciales.

Cela dit, on doit à Algol plusieurs innovations importantes, de telle
sorte qu'un grand nombre des langages qui verront le jour par la suite
seront considérés comme ses descendants; le poster
\link{http://www.oreilly.com/go/languageposter}{\emph{History of
    Programming Languages}} de O'Reilly Media illustre ceci à
merveille. \citet{Hoare:1973} a d'ailleurs cette jolie formule:
\begin{quote}
  Voici un langage très en avance sur son temps, il n'a pas seulement
  été une amélioration de ses prédécesseurs mais aussi une
  amélioration de presque tous ses successeurs.
\end{quote}

\begin{figure}[t]
  \centering
  \begin{minipage}{0.9\linewidth}
    \setkeys{Gin}{width=\textwidth}
    \includegraphics{images/standards} \\
    \footnotesize\sffamily%
    Tiré de \href{http://xkcd.com/927/}{XKCD.com}
  \end{minipage}
\end{figure}

\subsection{APL}
\label{sec:informatique:historique:apl}

Notre historique ne serait pas complet sans un mot sur \index{APL}APL
(\emph{A Programming Language}, qui l'aurait cru). Même s'il n'a
jamais connu une diffusion importante, ce langage conçu par Kenneth
Iverson autour de 1962 n'en a pas moins eu une influence considérable
sur la manière de penser et de représenter les opérations
mathématiques sur les tableaux à plusieurs dimensions.

Doté d'une large gamme de symboles pour représenter des opérations et
d'une syntaxe tout à fait particulière --- les expressions sont
exécutées de droite à gauche! --- l'APL est remarquablement concis et
puissant; voir la \autoref{fig:informatique:apl} pour un aperçu. Le
revers de la médaille et ce qui a assurément nui à son adoption
à large échelle, c'est la difficulté que l'on éprouve à relire le
code. Assez pour que d'aucuns qualifient l'APL de «langage à écriture
seulement».

APL a pendant longtemps été un langage très prisé par les actuaires,
aussi subsiste-t-il du code dans certaines compagnies d'assurance. Le
modèle de traitement des vecteurs, matrices et tableaux de l'APL a
servi d'inspiration pour la conception du langage S --- nous y
reviendrons au \autoref{chap:presentation}.

Le langage continue sa vie aujourd'hui principalement sous forme de son
successeur, \index{J}J.

\begin{figure}
  \centering
  \scalebox{0.4}{\includegraphics{images/APL_intro}}
  \caption[Opérations simples en APL.]{Opérations simples en APL. De
    haut en bas: génération des nombres de $1$ à $5$ et stockage dans
    la variable \code{x}; affichage du contenu de \code{x}; somme des
    éléments de \code{x}; moyenne des éléments de \code{x}. {\small Source:
    François-Dominique,
    \link{https://commons.wikimedia.org/w/index.php?curid=43207460}{Wikimedia
      Commons}, CC BY-SA 4.0}}
  \label{fig:informatique:apl}
\end{figure}

\subsection{C}
\label{sec:informatique:historique:c}

Le langage \index{C}C a été inventé en 1972 chez Bell Labs par Ken
Thompson et Dennis Ritchie afin de réécrire le système d'exploitation
UNIX (\autoref{sec:informatique:os:unix}). Il demeure beaucoup utilisé
pour la programmation système: le noyau de systèmes d'exploitation
comme Windows et Linux sont développés en grande partie en C.

Le C est un langage de programmation généraliste considéré, selon les
standards actuels, comme de bas niveau. Pour illustrer, l'utilisateur
doit programmer des traitements comme la libération de la mémoire, la
vérification de la validité des indices sur les tableaux, l'ouverture
et la fermeture des fichiers, etc.

Le langage demeure l'un des plus utilisés dans le monde et son
influence est considérable. De nombreux langages plus modernes comme
\index{C++@\Cpp}\Cpp, \index{C#@C\#}C\# et \index{Java}Java reprennent des
aspects de C. Le C est également beaucoup utilisé pour le calcul
numérique intensif, où il s'est en quelque sorte substitué à Fortran.
La plupart des progiciels scientifiques --- dont R, encore une fois
--- offrent la possibilité d'appeler du code C lorsque la rapidité de
calcul devient un enjeu. Le gain en temps d'exécution doit toutefois
être suffisamment grand pour compenser le temps de développement plus
long qu'exige le C par rapport à des langages de plus haut niveau.

\notebox{Dans l'ouvrage classique de \cite{KandR:1978}, le premier
  exemple d'un programme C affiche le message «\emph{hello, world}» à
  l'écran. Ça deviendra ensuite une tradition de démontrer le
  fonctionnement ou la syntaxe d'un langage avec cet exemple.}

\subsection{Autres jalons}
\label{sec:informatique:historique:autres}

Nous nous sommes attardés jusqu'ici à des langages de programmation
vieux de plus de 40 ans à cause de leur importance historique et parce
qu'ils sont toujours utilisés couramment. De nombreux autres langages
ont vu le jour depuis, si bien qu'ils se comptent aujourd'hui par
milliers. En voici quelques autres ayant occupé une place
prépondérante dans l'histoire.

\begin{itemize}
\item Comme son nom l'indique, \index{C++@\Cpp}\textbf{\Cpp} (Bjarne
  Stroustrup, 1980) est un dérivé du C qui lui ajoute, en autres
  choses, la programmation orientée objet. Certains considèrent que
  {\Cpp} devrait être le point d'entrée de toute personne voulant
  débuter à programmer avec des langages de la famille du C. Le code C
  est compatible avec le \Cpp.
\item Principal représentant de l'ère Internet des années 1990,
  \index{Java}\textbf{Java} (James Gosling, 1995) a été conçu pour
  que le code écrit dans ce langage puisse s'exécuter sur n'importe
  quelle plateforme informatique sans nécessiter une nouvelle
  compilation. Il est donc très populaire dans les applications web ou
  embarquées. Sa syntaxe est fortement inspirée du \Cpp.
\item \index{Visual Basic}\textbf{Visual Basic} (Microsoft, 1991) permet de développer des
  applications de manière interactive en disposant des composantes sur
  un canevas. Le langage est aujourd'hui discontinué, mais son dérivé
  \textbf{Visual Basic for Applications} (\index{VBA}VBA) demeure beaucoup
  utilisé dans les applications de la suite bureautique Office.
\item \index{Python}\textbf{Python} (Guido van Rossum, 1991) est un
  langage de haut niveau, orienté objet, multiplateforme et sous
  licence libre. C'est un des langages les plus utilisés aujourd'hui
  pour le calcul scientifique et l'analyse de données massives.
\end{itemize}

Certains langages de programmation ont une vocation généraliste,
certains visent des niches particulières, alors que d'autres cherchent
surtout à faire progresser l'état des connaissances dans la théorie
des langages. Quoi qu'il en soit, un langage de programmation demeure
un outil et, en informatique comme dans d'autres domaines, il convient
de choisir le meilleur outil pour accomplir une tâche donnée.

J'invite les lecteurs intéressés à en savoir davantage sur
l'histoire des langages de programmation en général, ou sur l'un ou
l'autre des langages mentionnés ci-dessus en particulier, à débuter
par les très complètes entrées de Wikipedia en
\link{https://fr.wikipedia.org/wiki/Histoire_des_langages_de_programmation}{français}
et en
\link{https://en.wikipedia.org/wiki/History_of_programming_languages}{anglais}.
\citet[chapitre~2]{Oualline:C:1997} propose également une excellente
introduction à la notion de programmation ainsi qu'une brève, mais
instructive, histoire du langage C.


\section{Langages compilés et interprétés}
\label{sec:informatique:compile_vs_interprete}

Tous les langages de programmation\footnote{%
  À l'exception du langage machine, mais rares sont les personnes qui
  souhaitent ne programmer qu'avec des $0$ et des $1$.} %
requièrent un traitement afin que les programmes écrits avec ceux-ci
puissent être exploités par un ordinateur. Il existe deux grandes
façons d'effectuer ce traitement: par compilation et par
interprétation.

Un \index{compilateur}compilateur est un programme informatique qui
transforme en langage machine le code source rédigé dans un langage
donné. On dit alors de ce langage qu'il est \index{compilé
  (langage)}\emph{compilé}. Le compilateur produit un fichier
informatique que l'ordinateur peut ensuite exécuter directement. Sur
la plateforme Windows, on reconnait notamment ces fichiers par leurs
extensions \code{.exe} ou \code{.dll}. Les noyaux d'à peu près toutes
les applications sur nos ordinateurs sont des fichiers compilés.

Un \index{interpréteur}interpréteur, quant à lui, est un programme qui
analyse, traduit et exécute le code source d'un programme
informatique, et ce, à chaque fois que le programme doit être exécuté.
Un langage qui nécessite l'intermédiaire d'un interpréteur est dit
\emph{interprété}. Un programme écrit dans un \index{interprété
  (langage)}langage interprété ne peut être distribué sans son
interpréteur.

Le traitement additionnel entre le code source et le langage machine
fait généralement en sorte que les langages interprétés sont plus
lents que les langages compilés à l'exécution. En revanche, une foule
de détails de mise en œuvre sont pris en charge par l'interpréteur, ce
qui réduit le temps de développement.


\section{Paradigmes de programmation}
\label{sec:informatique:paradigmes}

Un programme informatique est toujours écrit pour résoudre un problème
donné. Or, comme pour tout problème dans la vie, la solution à
celui-ci peut être formulée de différentes manières. Le style
fondamental avec lequel on exprime la solution est appelé le
paradigme de programmation.

Il existe plusieurs paradigmes de programmation. Nous nous
contenterons ici de se pencher sur seulement quatre.

\begin{description}
\item[Impératif] \index{paradigme!impératif} On indique à l'ordinateur
  les opérations à exécuter et l'ordre dans lequel les exécuter. C'est
  le paradigme le plus intuitif.
\item[Déclaratif] \index{paradigme!déclaratif} On indique cette fois à
  l'ordinateur ce que l'on souhaite obtenir comme résultat, mais sans
  préciser comment y parvenir. Il est laissé à la mise en œuvre du
  langage de déterminer la meilleure méthode de résolution du
  problème.

  Par exemple, pour extraire d'une base de données \code{etudiants} le
  prénom de toutes les personnes de plus de 25~ans, on écrira dans un
  langage déclaratif comme le SQL:
\begin{Schunk}
\begin{Verbatim}
SELECT prenom FROM etudiants WHERE age > 25
\end{Verbatim}
\end{Schunk}
  Nulle part n'est-il précisé comment le programme devrait procéder à
  l'extraction.
\item[Fonctionnel] \index{paradigme!fonctionnel} Un programme est une
  suite d'appels de fonctions, comme en mathématiques. L'exécution
  d'une fonction n'a pas d'impact sur les autres fonctions\footnote{%
    Autrement on dit qu'une fonction a un effet de bord (\emph{side
      effect}.} %
  Les opérations complexes sont réalisées en combinant les fonctions,
  de manière analogue à la composition de fonctions $g \circ f$.
\item[Orienté objet] \index{paradigme!orienté objet} Un programme est
  conçu comme un ensemble de blocs logiciels (les objets) qui
  interagissent entre eux. Une \emph{méthode} applique un traitement
  différent à un objet selon sa \emph{classe}. Ce paradigme est
  particulièrement utilisé dans les grands et complexes projets
  informatiques.
\end{description}

La plupart des langages d'usage courant combinent d'office, ou du
moins permettent de combiner, plusieurs paradigmes. Par exemple, le
langage {\Cpp} est à la fois un langage impératif et orienté objet.


\section{Sémantique et syntaxe}
\label{sec:informatique:semantique}

Il y a plusieurs parallèles à dresser entre les langages de
programmation et les langues parlées ou écrites: leur apprentissage
requiert de la pratique; on en connait jamais un trop grand nombre; le
premier est plus difficile à apprendre que les suivants.
L'informatique a également emprunté à la linguistique les notions de
\index{sémantique}sémantique et de \index{syntaxe}syntaxe.

La sémantique est l'étude de ce que signifie un message ou un
programme informatique, c'est-à-dire de ce qu'il transmet ou exécute.
La syntaxe, quant à elle, étudie la structure du message ou du
programme. En simplifiant, disons que, pour une langue donnée, un
dictionnaire permet de connaitre la sémantique, alors qu'une grammaire
décrit la syntaxe \citep{Hebenstreit:semantique}.

Par exemple, exprimons le message «J'ai soif» en anglais. Dire
«\emph{I am hungry}» relèverait d'une erreur de sémantique, puisque la
signification du message s'en trouve changée. En revanche, avec
«\emph{I have thirsty}» le bon message se rendra, mais peut-être pas
sans que l'erreur de syntaxe n'ait d'abord suscité une hésitation chez
l'interlocuteur.

Comme pour une langue, la maitrise d'un langage de programmation exige
de respecter à la fois les règles de sémantique et les règles de
syntaxe du langage. Le second volet est relativement facile: si le
programme compile ou s'exécute, c'est généralement que la syntaxe est
respectée. Le respect de la sémantique, ou de la manière propre d'un
langage d'exprimer des idées, demande plus d'efforts et de pratique.


\section{Systèmes d'exploitation}
\label{sec:informatique:os}

Le \index{système d'exploitation}système d'exploitation
(\emph{operating system}, OS) est un ensemble de programmes qui gère
les ressources matérielles et logicielles d'un ordinateur. Premier
programme exécuté par l'ordinateur lors de la mise en marche, le
système d'exploitation reçoit les demandes de ressources des
applications --- espace mémoire, unité de calcul, disque dur,
communication avec les périphériques, utilisation du réseau --- et les
alloue en fonction de l'état du système et des demandes concurrentes
des autres applications. Tous les logiciels applicatifs requièrent un
système d'exploitation pour fonctionner.

Il existe une grande variété de systèmes d'exploitation. Les plus
connus aujourd'hui sont, pour les ordinateurs personnels et les
serveurs, Windows de Microsoft, macOS de Apple et Linux, un logiciel
libre créé et toujours administré par Linus Torvalds. Pour les
appareils mobiles, deux systèmes d'exploitation se partagent
l'essentiel du marché: iOS et Android. Le premier est dérivé de macOS
et le second, de Linux.

Les systèmes macOS et Linux appartiennent à la famille des systèmes
Unix. Cela ne laisse donc en fait que deux grandes classes de systèmes
d'exploitation couramment utilisés: Windows et Unix.

\subsection{Windows}
\label{sec:informatique:os:windows}

Le système d'exploitation \index{Windows}Windows de Microsoft équipe
près de 90~\% des ordinateurs personnels dans le monde. Dans les
circonstances, rares sont les personnes qui n'ont jamais utilisé le
système, ne serait-ce qu'à petites doses.

À l'origine, en 1985, Windows n'était qu'une interface graphique
superposée au véritable système d'exploitation des ordinateurs de type
«compatibles IBM PC», le DOS. Les versions grand public depuis
Windows~2000 sont plutôt basées sur le noyau de Windows~NT, un système
conçu à partir de zéro et lancé au milieu des années 1990 en tant que
système pour les entreprises.

Le plus grand avantage de Windows est son ubiquité. À peu près toutes
les applications, sauf peut-être celles s'adressant à un segment de
marché très précis, sont disponibles sous Windows, en code natif, via
un port depuis Unix, ou encore par l'intermédiaire d'une couche
logicielle d'émulation comme Cygwin ou \index{MinGW}MinGW.

Un système d'exploitation multiutilisateur prend pour acquis que
plusieurs utilisateurs se partageront les ressources de l'ordinateur,
simultanément lorsqu'il s'agit d'un serveur, ou les uns après les
autres dans le cas d'un ordinateur personnel. Dans un tel système, on
établit des règles strictes d'accès aux ressources du système (seul
l'administrateur peut installer ou supprimer des applications) et
d'accès aux ressources des autres utilisateurs (Marianne n'a pas
accès aux fichiers d'Alexandre et vice-versa).

Les versions de Windows multiutilisateurs sont apparues relativement
tard sur le marché grand public, et ce, sans que le paradigme ne soit
imposé aux utilisateurs. Résultat: la plupart des utilisateurs de
Windows s'octroient à la ronde les droits d'administrateur et, pire,
plusieurs applications nécessitent toujours un accès total au système.
C'est par ces failles que se propagent virus et autres logiciels
malveillants sur la plateforme.

Les systèmes Windows sont livrés avec une trousse de développement
très restreinte. Les programmeurs doivent donc installer les outils
nécessaires, souvent sous forme de suites logicielles monolithiques,
ce qui entraine dédoublements et, parfois, conflits.

\subsection{Unix}
\label{sec:informatique:os:unix}

Le terme \index{Unix}\emph{Unix} --- ou UNIX qui, en majuscules, est
une marque déposée de \link{http://www.opengroup.org/unix}{Open Group}
--- recoupe une famille de systèmes d'exploitation dérivant tous du
premier système Unix développé chez Bell Labs dans les années 1970 par
Kenneth Thompson, Dennis Ritchie et plusieurs autres.

Les systèmes Unix partagent un certain nombre de caractéristiques
communes. D'abord, ils sont intrinsèquement multitâches et
multiutilisateurs. Ensuite, la
\Index{Unix!philosophie}\link{https://fr.wikipedia.org/wiki/Philosophie_d\%27Unix}{philosophie
  Unix} commande la modularité: le système d'exploitation fournit aux
utilisateurs (via un interpréteur de commandes, ou \emph{shell}) et
aux applications une multitude de petits outils très spécialisés qui
ne réalisent à peu près qu'une seule tâche. Le système rend ensuite
simple, via un mécanisme dit de transfert de données («tuyau»,
\emph{pipe}) de les combiner pour effectuer des opérations plus
complexes.

Par exemple, pour extraire le mot \emph{tuyau} du code source
contenant le précédent paragraphe, nous pourrions utiliser l'outil
\code{grep} pour extraire la ligne où se trouve le mot, puis l'outil
\code{cut} pour isoler le mot qui nous intéresse et, enfin, l'outil
\code{tr} pour supprimer la parenthèse ouvrante, les guillemets et la
virgule autour du mot.\footnote{%
  Dans la mesure où le mot que l'on recherche est le quatrième de la
  ligne dans le fichier source \code{informatique.tex}, la commande
  résultante est: \code{grep "«tuyau»" informatique.tex | cut -d " "
    -f 4 | tr -d \bs(«»,}\,. Le caractère \code{|} est l'opérateur de
  transfert de données.}

À l'exception notable des Mac, \index{Unix}Unix demeure assez peu répandu sur les
ordinateurs personnels. En revanche le système d'exploitation, en
particulier sa variante libre Linux, équipe l'immense majorité des
stations de travail, serveurs et supercalculateurs. De plus, bon
nombre d'applications scientifiques et de langages de programmation
sont d'abord développés sur des systèmes Unix. Pour ces
raisons, j'estime utile pour des programmeurs de connaitre les
quelques «Unix-ismes» ci-dessous.

\begin{itemize}
\item La barre oblique (\code{/}) est utilisée pour séparer les
  dossiers dans les chemins d'accès aux fichiers.
\item Tout utilisateur dispose d'un espace disque réservé identifié
  comme le \index{répertoire personnel}\emph{répertoire personnel} ou
  le \index{dossier de départ|see{répertoire personnel}}\emph{dossier
    de départ} (\emph{home directory}). Ce répertoire est situé dans
  \code{/Users} sous macOS et habituellement dans \code{/home} sous
  Linux.
\item À la ligne de commande et dans la configuration de plusieurs
  applications, le caractère \,\verb=~=\, représente le répertoire
  personnel.
\item Toujours à la ligne de commande, il est possible d'appuyer sur
  \code{TAB} pour compléter les noms de fichiers, de répertoires ou de
  commandes.
\item Plusieurs applications permettent de stocker des options de
  configuration sous forme de texte brut dans des fichiers dont le nom
  débute généralement par un point «\verb=.=»\,.
\end{itemize}
La section suivante fournit des détails additionnels sur le système de
fichiers de Unix.


\section{Systèmes de fichiers}
\label{sec:informatique:fs}

Le \index{système de fichiers}système de fichiers d'un ordinateur
contrôle l'inscription, l'organisation et la récupération des données
sur une unité de stockage, souvent un disque dur.

Nous n'entrerons pas dans les détails techniques des systèmes de
fichiers. Seulement, les programmeurs doivent être familiers avec
quelques grands principes de l'organisation des données dans un
ordinateur.

Les fichiers sont regroupés dans des répertoires. Ceux-ci contiennent
eux-mêmes des fichiers ou des sous-répertoires. Ce schéma se répète
sans limite pratique. Il en résulte une classification sous forme
d'arbre dont la \emph{racine} est le répertoire contenant
l'intégralité des fichiers. La \autoref{fig:informatique:fs} présente
des extraits des arbres de fichiers usuels de Windows et de macOS.

\begin{figure}
  \begin{minipage}[t]{0.45\linewidth}
    \dirtree{%
      .1 \code{C:\bs}.
      .2 \code{Program Files\bs}.
      .3 \code{Internet Explorer\bs}.
      .3 \code{Windows NT\bs}.
      .2 \code{Users\bs}.
      .3 \code{Alexandre\bs}.
      .4 \code{Desktop\bs}.
      .4 \code{Documents\bs}.
      .5 \code{synthese-1.doc}.
      .2 \code{Windows\bs}.
      .3 \code{explorer.exe}.
      .3 \code{System32\bs}.
      .4 \code{wininit.exe}.
    }
  \end{minipage}
  \hfill
  \begin{minipage}[t]{0.45\linewidth}
    \dirtree{%
      .1 \code{/}.
      .2 \code{/Applications}.
      .3 \code{/Mail.app}.
      .3 \code{/Preview.app}.
      .2 \code{/Users}.
      .3 \code{/Marianne}.
      .4 \code{/Desktop}.
      .5 \code{freestyle.txt}.
      .4 \code{/Documents}.
      .5 \code{/Cours}.
      .2 \code{/System}.
      .2 \code{/bin}.
      .3 \code{bash}.
      .2 \code{/usr}.
      .3 \code{/bin}.
      .4 \code{grep}.
    }
  \end{minipage}
  \caption[Extraits de la hiérarchie des systèmes de fichiers]{%
    Extraits des arbres de fichiers de Windows~7 (à gauche) et de
    macOS (à droite)}
  \label{fig:informatique:fs}
\end{figure}

\subsection{Particularités du système de fichiers de Windows}
\label{sec:informatique:fs:windows}

Sous Windows, chaque lecteur physique ou lecteur réseau dispose de son
propre arbre de fichiers. La racine est identifiée par une lettre. Le
premier disque dur est \code{C}\footnote{%
  La nomenclature des lecteurs date d'une époque où les ordinateurs
  personnels n'étaient pas tous équipés d'un disque dur, mais plutôt
  d'un ou deux lecteurs de disquettes. Ceux-ci étaient identifiés par
  les lettres \code{A} et \code{B}.}. %
Si le système comporte plus d'un disque, l'utilisateur doit savoir sur
quel disque retrouver ou enregistrer ses données.

Le système de fichiers de Windows est insensible à la casse,
c'est-à-dire qu'il ne fait aucune distinction entre \code{Program
  Files}, \code{program files} et \code{PROGRAM FILES}. Dans les
chemins d'accès (voir ci-dessous), la barre oblique inversée
{\bs} sépare les noms de répertoires.

\subsection{Particularités du système de fichiers de Unix}
\label{sec:informatique:fs:unix}

La racine du système de fichiers est toujours \code{/} sous
\index{Unix}Unix. Les disques, physiques ou réseau, sont accessibles
par divers \emph{points de montage} dans le système de fichiers.

Pour illustrer, supposons qu'un ordinateur dispose de deux disques
durs; le premier est réservé au système d'exploitation et aux
applications partagées, alors que le second renferme les fichiers
personnels des utilisateurs. Dans un tel scénario, le premier disque
sera monté sur le répertoire \code{/} et le second, sur le répertoire
\code{/home}. Le système de fichier dirigera automatiquement les
requêtes vers le bon disque. L'utilisateur n'a jamais à se préoccuper
de l'organisation physique des disques.

\osxbox{Sous macOS, les clés USB, disques durs externes ou disques réseau sont
  montés dans le répertoire \code{/Volumes}. Le répertoire
  n'existe pas lorsqu'aucun disque externe n'est connecté.}

Le système de fichiers sous Unix est généralement sensible à la casse.
La barre oblique \code{/} sépare les noms de répertoires dans les
chemins d'accès.

Les fichiers dont le nom débute par un point «\verb=.=» sont des
fichiers cachés. À moins de demander explicitement de les afficher,
ils n'apparaissent donc pas dans la liste des fichiers à la ligne de
commande ou dans les interfaces graphiques.

\subsection{Chemin d'accès}
\label{sec:informatique:fs:path}

Le \index{chemin d'accès}chemin d'accès (\emph{path}) d'un fichier ou
d'un répertoire décrit la position de la ressource dans le système de
fichiers. Un chemin d'accès peut être absolu ou relatif.
\begin{description}
\item[Chemin absolu] \index{chemin d'accès!absolu} La position d'un
  fichier est décrite à partir de la racine, de telle sorte que le
  chemin d'accès demeure valide depuis n'importe quel point dans le
  système de fichier.

  Dans l'extrait de système de fichier Windows de la
  \autoref{fig:informatique:fs}, le chemin d'accès absolu vers le
  fichier \code{synthese-1.doc} est:
\begin{Schunk}
\begin{Verbatim}
C:\Users\Alexandre\Documents\synthese-1.doc
\end{Verbatim}
\end{Schunk}
\item[Chemin relatif] \index{chemin d'accès!relatif} La position d'un
  fichier est donnée à partir d'un endroit précis dans le système de
  fichier autre que la racine. Le chemin dépend donc du répertoire
  courant. On a recours au nom fictif «\verb=..=» pour identifier le
  répertoire parent (un niveau supérieur dans l'arbre des fichiers).

  Par exemple, si le répertoire courant de Marianne est
  \code{Documents/Cours} dans le système de fichier Unix de la
  \autoref{fig:informatique:fs}, alors le chemin d'accès relatif vers
  le fichier \code{freestyle.txt} est:
\begin{Schunk}
\begin{Verbatim}
../../Desktop/freestyle.txt
\end{Verbatim}
\end{Schunk}
\end{description}

La notion de chemin d'accès fait également référence à la liste des
répertoires dans lesquels le système d'exploitation recherche une
application lorsqu'elle est appelée. Cette liste est conservée dans
une variable d'environnement nommée \verb=%PATH%= sous Windows
et \verb=$PATH= sous \index{Unix}Unix. Il est hors de la portée de ce
document d'expliquer comment modifier la variable d'environnement. La
réponse se trouve à une requête près dans un moteur de recherche.


\section{Ligne de commande}
\label{sec:informatique:cli}

Une interface en \index{ligne de commande}ligne de commande
(\emph{command line interface}, CLI), ou tout simplement \emph{ligne
  de commande}, est un mode d'interaction avec un programme
informatique dans lequel l'utilisateur dicte les commandes et reçoit
les réponses de l'ordinateur en mode texte. Le programme qui gère
cette interface est un \emph{interpréteur de commandes}, parfois aussi
appelé (à tort) \emph{terminal} ou \emph{shell}. Lorsque
l'interpréteur est prêt à recevoir une commande, il l'indique par une
\emph{invite de commande} (\emph{command prompt}).

La ligne de commande est la plus ancienne des interfaces avec les
ordinateurs. Si les interfaces graphiques ont à plusieurs égards
grandement facilité l'interaction homme--machine, elles n'ont pas pour
autant fait disparaitre ou rendu obsolète la ligne de commande.
D'abord, celle-ci demeure parfois la seule --- et souvent la plus
rapide --- interface pour réaliser certaines tâches sur un ordinateur
muni d'une interface graphique. Ensuite, certains ordinateurs, comme
les serveurs ou les supercalculateurs, n'offrent souvent aucune
interface graphique.

\subsection{Ligne de commande Windows}
\label{sec:informatique:cli:windows}

L'interpréteur de commandes standard sous Windows est
\index{cmd@\code{cmd.exe}}\code{cmd.exe}. On le trouve dans le groupe
de programmes \code{Accessoires} sous le nom \code{Invite de
  commandes}. L'invite de commande par défaut est \verb=C:\>=. À la
gauche du symbole \verb=>= se trouve le chemin d'accès du répertoire
courant. Au lancement de l'interpréteur, celui-ci est donc la racine
du système de fichiers.

Les programmeurs qui souhaitent bénéficier de certains outils
\index{Unix}Unix sous Windows peuvent installer les couches
logicielles \index{MinGW}MinGW et \index{MSYS}MSYS --- je recommande à
cet égard la distribution
\link{http://www.msys2.org}{MSYS2}\index{MSYS2}. Le programme
\code{MSYS} fournit un interpréteur de commandes \index{Bash}Bash
\citep{bash} comme en standard sur les systèmes Unix. Un interpréteur
Bash est également fourni avec \index{Git}
\link{https://git-scm.com/downloads}{Git for Windows}: le programme
est d'ailleurs nommé \index{Git~Bash}Git~Bash.

À la ligne de commande \index{Unix}Unix sous Windows, on identifie la
racine \code{C:\bs} du système de fichier d'un disque par \code{/c/}.

\subsection{Ligne de commande macOS}
\label{sec:informatique:cli:macos}

Sous macOS, l'application de ligne de commande se nomme
\index{Terminal}Terminal. Elle se trouve dans le sous-dossier
Utilitaires du dossier Applications. L'interpréteur de commandes est
\index{Bash}Bash. L'invite de commande par défaut est le symbole
\verb=$=, habituellement précédé du nom de l'ordinateur, du répertoire
courant et du nom de l'utilisateur. Le répertoire courant au lancement
de l'interpréteur est le répertoire personnel \,\verb=~=\, de
l'utilisateur.

\tipbox{Explorez les préférences de l'application
  \index{Terminal}Terminal pour configurer la police de caractères et
  les couleurs afin de rendre la ligne de commande plus agréable à
  utiliser.}

\subsection{Commandes essentielles}
\label{sec:informatique:cli:commandes}

Les programmeurs doivent connaitre les rudiments de la ligne de
commande. Cela dit, comme nous ne pouvons nous étendre sur le sujet,
limiterons la présentation aux commandes ci-dessous qui
permettent de se déplacer dans le système de fichiers et d'afficher
la liste des fichiers d'un répertoire.

\importantbox{Dans les exemples ci-dessous et pour le reste du
  document, l'invite de commande du système d'exploitation sera
  représentée par le symbole \code{\$} précédé du nom du répertoire
  courant (lorsque cela s'avère pertinent).}

\begin{ttscript}{pwd}
\item[\Icode{cd}] Change le répertoire courant pour
  celui donné en argument. Cet argument peut être un chemin d'accès
  absolu ou relatif (\autoref{sec:informatique:fs:path}). Le nom
  fictif «\verb=..=» identifie le parent du répertoire courant.

  Lorsque utilisée sans argument, la commande affiche le chemin
  d'accès absolu du répertoire courant avec \code{cmd.exe}, ou ramène
  directement l'invite de commande au répertoire personnel de
  l'utilisateur avec Bash.
  \begin{Schunk}
\begin{Verbatim}
~$ cd Desktop/
/Users/vincent/Desktop
~/Desktop$ cd ~/Documents/cours/
~/Documents/cours$ cd
~$
\end{Verbatim}
  \end{Schunk}
\item[\Icode{pwd}] (\index{Bash}Bash seulement) Affiche le chemin
  d'accès absolu du répertoire courant.
  \begin{Schunk}
\begin{Verbatim}
~$ pwd
/Users/vincent
\end{Verbatim}
  \end{Schunk} %$
\item[\Icode{dir}] (\index{cmd@\code{cmd.exe}}\code{cmd.exe}
  seulement) Affiche les fichiers du répertoire donné en argument, ou
  les fichiers du répertoire courant sans aucun argument.
\item[\code{ls}] \Index{ls@\code{ls} (Bash)} (\index{Bash}Bash
  seulement) Affiche les fichiers du répertoire donné en argument, ou
  les fichiers du répertoire courant sans aucun argument.
  \begin{Schunk}
\begin{Verbatim}[commandchars=\\\{\}]
~$ ls
Desktop
Documents
Downloads
\meta{...}
\end{Verbatim}
  \end{Schunk} %$
\end{ttscript}

\tipbox{Les systèmes d'exploitation affichent certains répertoires
  (\code{Bureau} ou \code{Musique}, par exemple) dans la langue de
  l'interface plutôt que sous leur vrai nom anglais. La commande
  \code{ls} affiche toujours le véritable nom des répertoires.}

\section{Exercices}
\label{operateurs:exercices}

\Opensolutionfile{solutions}[solutions-informatique]

\begin{Filesave}{solutions}
\section*{Chapitre \ref*{chap:informatique}}
\addcontentsline{toc}{section}{Chapitre \protect\ref*{chap:informatique}}

\end{Filesave}

\begin{exercice}[nosol]
  Des nouveaux langages de programmation apparaissent régulièrement.
  Deux exemples récents provenant de géants de l'industrie sont
  \link{https://swift.org}{Swift}, de Apple, et
  \link{https://golang.org}{Go}, de Google. À partir des informations
  glanées sur les sites Internet des langages et dans les pages
  Wikipedia qui leurs sont consacrées, retracer les grands langages
  ayant servi comme sources d'inspiration à \index{Swift}Swift et à
  \index{Go}Go.
\end{exercice}

\begin{exercice}
  Soit $A$, $B$, $C$ et $D$ quatre nombres. L'opération en notation
  infixée
  \begin{equation*}
    A \times B + C \div D
  \end{equation*}
  s'écrit
  \begin{equation*}
    +\, \times\, A\; B\, \div\, C\; D
  \end{equation*}
  en notation préfixée et
  \begin{equation*}
    A\; B\, \times\, C\; D\, \div +
  \end{equation*}
  en notation suffixée.
  \begin{enumerate}
  \item Exprimer en notations préfixée et suffixée l'opération en
    notation infixée suivante:
    \begin{equation*}
      A \times (B + C) \div D.
    \end{equation*}
  \item Les opérations suivantes exprimées en notation préfixée et
    suffixée, dans l'ordre, effectuent la même opération arithmétique.
    \begin{gather*}
      \times\, A\, +\, B\, \div\, C\; D \\
      A\; B\; C\; D\, \div +\, \times
    \end{gather*}
    Exprimer cette opération en notation infixée.
  \end{enumerate}
  \begin{sol}
    \begin{enumerate}
    \item $\div \times\, A\, +\, B\; C\; D$ en notation préfixée;
      $A\; B\; C\, +\, \times\, D\, \div$ en notation suffixée.
    \item $A \times (B + C \div D)$
    \end{enumerate}
  \end{sol}
\end{exercice}

\begin{exercice}
  Démarrer un interpréteur de commandes Bash. Si le système
  d'exploitation de votre ordinateur est Windows, vous pouvez utiliser
  la ligne de commande \index{Git~Bash}Git~Bash de
  \link{https://git-scm.com/downloads}{\emph{Git for Windows}} ou la
  ligne de commande MSYS de
  \index{MSYS2}\link{http://www.msys2.org}{MSYS2}.
  \begin{enumerate}
  \item Afficher le répertoire courant. Vérifier qu'il correspond à
    celui mentionné dans l'invite de commande. Localiser ce répertoire
    dans l'interface graphique de votre système d'exploitation (Finder
    sous macOS, Windows~Explorer sous Windows).
  \item Afficher la liste des fichiers du répertoire courant.
  \item Choisir un sous-répertoire du répertoire courant que vous
    savez contenir lui-même un sous-répertoire (par exemple:
    \code{Documents/Cours}. Afficher, sans d'abord s'y déplacer, la
    liste des fichiers du premier sous-répertoire (\code{Documents}),
    puis celle du second (\code{Cours}).
  \item Faire du sous-répertoire utilisé précédemment le répertoire
    courant.
  \item Afficher une liste détaillée des fichiers du nouveau
    répertoire courant avec la commande \verb=ls -l=.
  \item Descendre d'un autre niveau dans l'arbre des fichiers.
  \item Revenir au répertoire personnel avec une seule commande.
  \item Afficher la liste de tous les fichiers du répertoire
    personnel, y compris les fichiers cachés, avec la commande
    \verb=ls -a=.
  \item Afficher la liste détaillée de tous fichiers du répertoire
    personnel avec la commande \verb=ls -la=.
  \item Afficher la liste des répertoires dans lesquels le système
    d'exploitation recherche des applications avec la commande
    \verb=echo $PATH=.
  \end{enumerate}
  \begin{sol}
    Vous trouverez ci-dessous une transcription (partielle) de la
    séance de travail à effectuer depuis une ligne de commande Bash.
    \begin{Schunk}
\begin{Verbatim}[commandchars=\\\{\}]
\textbf{~$} pwd
/Users/vincent
\textbf{~$} ls
Desktop           Music        emacs-modified-extensions
Documents         Pictures     emacs-modified-macos
Downloads         Public       emacs-modified-windows
\meta{...}
\textbf{~$} ls Documents/
Spoon-Knife
actexam
actuar
actuarialangle
actuarialsymbol
\meta{...}
\textbf{~$} ls Documents/cours/
ACT-2002        ACT-2005        ACT-2008        IFT-1902
\textbf{~$} cd Documents/
/Users/vincent/Documents
\textbf{~/Documents$} ls -l
drwxr-xr-x   6 vincent staff 204 29 nov 14:40 Spoon-Knife
drwxr-xr-x   6 vincent staff 204 21 nov  2016 actexam
drwxr-xr-x  10 vincent staff 340  1 sep 09:56 actuar
\meta{...}
\textbf{~/Documents$} cd cours/
/Users/vincent/Documents/cours
\textbf{~/Documents/cours$} cd
\textbf{~$} ls -a
.                     .bashrc           .profile
..                    .cache            .rstudio-desktop
.CFUserTextEncoding   .cups             .subversion
\meta{...}
\textbf{~$} ls -la
total 584
drwxr-xr-x@ 71 vincent staff  2414  4 déc 21:03 .
drwxr-xr-x   5 root    admin   170 19 oct 04:13 ..
drwxr-xr-x   4 vincent staff   136 10 mai  2017 .R
-rw-r--r--   1 vincent staff    63  5 jul  2013 .Renviron
-rw-r--r--   1 vincent staff 23185  9 nov 09:13 .Rhistory
-rw-r--r--   1 vincent staff    71  4 jan  2016 .Rprofile
\meta{...}
\textbf{~$} echo PATH
/opt/local/bin:/opt/local/sbin:~/bin:/usr/local/bin:
/usr/bin:/bin:/usr/sbin:/sbin:/Library/TeX/texbin
\end{Verbatim}
    \end{Schunk}
  \end{sol}
\end{exercice}

\Closesolutionfile{solutions}

%%% Local Variables:
%%% mode: latex
%%% TeX-engine: xetex
%%% TeX-master: "programmer-avec-r"
%%% coding: utf-8
%%% End:
