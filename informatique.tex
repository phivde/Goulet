%%% Copyright (C) 2017 Vincent Goulet
%%%
%%% Ce fichier fait partie du projet «Programmer avec R»
%%% http://github.com/vigou3/programmer-avec-r
%%%
%%% Cette création est mise à disposition selon le contrat
%%% Attribution-Partage dans les mêmes conditions 4.0
%%% International de Creative Commons.
%%% http://creativecommons.org/licenses/by-sa/4.0/

\chapter{Éléments d'informatique pour programmeurs}
\label{chap:informatique}

Intro du chapitre

\section{Bref historique des langages de programmation}
\label{sec:informatique:historique}

Ada Lovelace (1815--1852) est généralement reconnue comme la première
auteure d'un algorithme et de ce que l'on appelle aujourd'hui un
programme informatique. Le langage Ada, conçu en réponse à un cahier
de charges du département de la Défense des États-Unis au début des
années 1980, a d'ailleurs été nommé en son honneur.

Franchissons d'un bond les premiers jours de l'informatique et de la
programmation\footnote{%
  Les lecteurs intéressés à en apprendre davantage sur le sujet
  pourront débuter par les entrées de Wikipedia
  (\link{https://fr.wikipedia.org/wiki/Histoire_des_langages_de_programmation}{français},
  \link{https://en.wikipedia.org/wiki/History_of_programming_languages}{anglais}).} %
pour arriver aux ordinateurs électriques modernes, dans les années
1940. On programme généralement ceux-ci en \emph{assembleur}, un
langage de très bas niveau facilement interprétable par la machine,
mais difficile à lire par des humains; voir la
\autoref{fig:informatique:assembleur}.

\begin{figure}
  \centering
  \includegraphics[trim=0 475 0 0, clip=true]{Motorola_6800_Assembly_Language.png}
  \caption[Programme assembleur pour un microprocesseur 8~bits
  Motorola 6800.]{Extrait d'un programme en assembleur pour un
    microprocesseur 8~bits Motorola 6800. Source:
    \link{https://commons.wikimedia.org/wiki/File\%3AMotorola_6800_Assembly_Language.png}{Wikimedia Commons}}
  \label{fig:informatique:assembleur}
\end{figure}

Premiers langages créés pour transmettre des instructions à un
ordinateur à partir des années 1950. À l'origine intimement liés à
l'architecture propre d'un ordinateur; autrement dit, à chaque type
d'ordinateur son langage de programmation.

Plus vieux langages encore en activité: COBOL (1959), FORTRAN (1954),
LISP (1958)

Certaines caractéristiques --- ou exigences --- des plus vieux
langages proviennent de contraintes liées au support physique utilisé
à l'époque de leur création: les cartes perforées.

Empreinte de Fortran en calcul scientifique toujours importante,
notamment via BLAS et LAPACK

Algol Fin des années 1960 publication, by a committee of American and
European computer scientists, "a new language for algorithms"; the
ALGOL 60 Report (the "ALGOrithmic Language"). Velléité de
standardisation des langages de programmation; évidemment un échec;
néanmoins plusieurs innovations importantes qui feront en sorte qu'un
grand nombre des langages qui verront le jour par la suite seront
considérés comme des descendants d'Algol.

APL (1962) longtemps un langage très prisé par les actuaires; probablement
encore du code APL dans certaines compagnies d'assurance; inspiration
pour S et R; continue sa vie aujourd'hui surtout sous J.

Lisp langage beaucoup utilisé en intelligence artificielle;
particulièrement élégant, sauf peut-être pour la syntaxe. Citation:
\emph{}

"Lisp is worth learning for the profound enlightenment experience you will have when you finally get it; that experience will make you a better programmer for the rest of your days, even if you never actually use Lisp itself a lot."
- Eric Raymond, "How to Become a Hacker"
http://www.paulgraham.com/quotes.html

Greenspun's tenth rule of programming is an aphorism in computer programming and especially programming language circles that states:[1][2]

Any sufficiently complicated C or Fortran program contains an ad-hoc,
informally-specified, bug-ridden, slow implementation of half of
Common Lisp.
Version courte: Ceux qui ne connaissent pas le Lisp sont condamnés à
le réinventer
https://en.wikipedia.org/wiki/Greenspun%27s_tenth_rule


C Kernighan et Ritchie, Bell Labs -> John Chambers et S; Unix;
toujours beaucoup utilisé pour le calcul numérique intensif; évolution
vers C++

Python (1991)
\url{https://en.wikipedia.org/wiki/Python_(programming_language)}

Infix [infixée], Postfix [suffixée] and Prefix [préfixée] notation (encadré?)
\url{http://www.cs.man.ac.uk/~pjj/cs212/fix.html}






\section{Syntaxe et sémantique}
\label{sec:informatique:syntaxe}

\section{Paradigmes de programmation}
\label{sec:informatique:paradigmes}

\section{Systèmes d'exploitation}
\label{sec:informatique:os}



%%% Local Variables:
%%% mode: latex
%%% TeX-engine: xetex
%%% TeX-master: "programmer-avec-r"
%%% End:
