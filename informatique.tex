%%% Copyright (C) 2017 Vincent Goulet
%%%
%%% Ce fichier fait partie du projet «Programmer avec R»
%%% http://github.com/vigou3/programmer-avec-r
%%%
%%% Cette création est mise à disposition selon le contrat
%%% Attribution-Partage dans les mêmes conditions 4.0
%%% International de Creative Commons.
%%% http://creativecommons.org/licenses/by-sa/4.0/

\chapter{Éléments d'informatique pour programmeurs}
\label{chap:informatique}

Intro du chapitre

\section{Bref historique des langages de programmation}
\label{sec:informatique:historique}

Ada Lovelace (1815--1852) est généralement reconnue comme la première
auteure d'un algorithme et de ce que l'on appelle aujourd'hui un
programme informatique. C'est en son honneur qu'été nommé le langage
Ada conçu en réponse à un cahier de charges du département de la
Défense des États-Unis au début des années 1980.

Franchissons d'un bond les premiers jours de l'informatique et de la
programmation\footnote{%
  Les lecteurs intéressés à en apprendre davantage sur le sujet
  pourront débuter par les entrées de Wikipedia
  (\link{https://fr.wikipedia.org/wiki/Histoire_des_langages_de_programmation}{français},
  \link{https://en.wikipedia.org/wiki/History_of_programming_languages}{anglais}).} %
pour arriver aux ordinateurs électriques modernes, dans les années
1940. On programme alors généralement ceux-ci en \emph{assembleur}, un
langage de très bas niveau facilement interprétable par la machine,
mais difficile à lire par des humains; voir la
\autoref{fig:informatique:assembleur}.

\begin{figure}
  \centering
  \includegraphics[trim=0 475 0 0, clip=true]{Motorola_6800_Assembly_Language.png}
  \caption[Programme en assembleur pour un microprocesseur 8~bits
  Motorola 6800.]{Extrait d'un programme en assembleur pour un
    microprocesseur 8~bits Motorola 6800. {\small Source:
      \link{https://commons.wikimedia.org/wiki/File\%3AMotorola_6800_Assembly_Language.png}{Wikimedia
        Commons}}}
  \label{fig:informatique:assembleur}
\end{figure}

Les premiers langages créés pour transmettre des instructions à un
ordinateur apparaissent dans les années 1950. Ils sont à l'origine
intimement liés à l'architecture d'un ordinateur: à chaque type
d'ordinateur son langage de programmation. Certaines contraintes ---
ou exigences --- des langages de l'époque proviennent aussi du support
physique alors utilisé pour stocker les programmes: les cartes
perforées.

\subsection{Fortran}
\label{sec:informatique:historique:fortran}

En 1954, l'ingénieur de IBM John Backus publie les spécifications du
langage FORTRAN (\emph{FORmula TRANslating System}). Le premier
compilateur voit le jour deux ans plus tard.
\begin{itemize}
\item Fortran (avec des minuscules à partir de 1977) deviendra
  rapidement le langage standard dans le calcul scientifique.
\item Plus d'un demi-siècle plus tard, l'empreinte de Fortran demeure
  importante, notamment grâce aux bibliothèques d'algèbre linéaire
  BLAS\footnote{%
    \emph{Basic Linear Algebra Subprograms};
    \link{http://www.netlib.org/blas}{}.} %
  et LAPACK\footnote{%
    \emph{Linear Algebra PACKage};
    \link{http://www.netlib.org/lapack}{}.} %
  auxquelles ont recours la plupart des progiciels scientifiques, dont
  R.
\item Le langage est toujours utilisé en calcul haute performance et
  pour mesurer le rendement des superordinateurs.
\end{itemize}

\begin{figure}[t]
  \notebox{Dans le très recommandable film Les figures de l'ombre
    (\emph{Hidden Figures}, 2016), une des héroïnes entreprend de
    s'attaquer à la programmation des nouveaux ordinateurs de la NASA.
    On peut alors voir qu'elle apprend le Fortran.}
\end{figure}

\subsection{Lisp}
\label{sec:informatique:historique:lisp}

Le deuxième langage le plus ancien toujours largement diffusé est Lisp
(\emph{LISt Processing}). Créé par John McCarthy en 1958 en tant que
modèle pratique pour représenter des programmes, le Lisp est devenu le
langage de choix pour la recherche et les applications en intelligence
artificielle. Le terme Lisp désigne aujourd'hui une famille de
langages comprenant de nombreux dialectes, dont Common Lisp, Scheme et
Emacs Lisp.

\begin{itemize}
\item Le Lisp se distingue en outre par une syntaxe simple en notation
  préfixée (voir encadré), le support pour la programmation
  fonctionnelle (\autoref{sec:informatique:paradigmes}) et la faculté
  de manipuler le code source en tant que structure de données.
\item Autre trait distinctif: la syntaxe du Lisp fait un usage
  immodéré des parenthèses.
\item Le Lisp est entouré d'une aura de beauté et d'élégance dont peu
  d'autres langages peuvent se targuer. Citons Eric Raymond dans
  \link{http://www.catb.org/esr/faqs/hacker-howto.html}{\emph{How to
      Become a Hacker}}:
  \begin{quote}
    Il faut apprendre le Lisp pour l'extraordinaire expérience d'éveil
    [\emph{enlightenment experience}] que procure le fait de finalement
    le comprendre; cette expérience fera de vous un meilleur
    programmeur pour toujours, même si vous n'avez plus vraiment à
    utiliser le Lisp.
  \end{quote}
\item Pour illustrer encore davantage la place toute particulière
  qu'occupe le Lisp en programmation, mentionnons également
  l'aphorisme selon lequel «ceux qui ne connaissent pas le Lisp sont
  condamnés à le réinventer», d'une certaine façon la version courte
  de la célèbre
  \link{https://en.wikipedia.org/wiki/Greenspun\%27s_tenth_rule}{\emph{Greenspun's
      tenth rule of programming}}\footnote{%
    Fait à noter, il n'y a pas d'autres lois que la dixième, de
    l'\link{http://philip.greenspun.com/bboard/q-and-a-fetch-msg?msg_id=000tgU}{aveu
      même de l'auteur}.}: %
  \begin{quote}
    Tout programme en C ou en Fortran suffisamment complexe contient
    une implémentation mal spécifiée, pleine de bogues et lente de la
    moitié de Common Lisp.
  \end{quote}
\end{itemize}

\begin{figure}[t]
  \setlength{\FrameRule}{1pt}
  \begin{emphbox}{\mdseries Notations infixée, préfixée et suffixée}
    Les notations infixée (\emph{infix}), préfixée (\emph{prefix}) et
    suffixée (\emph{postfix}) sont trois manières différentes, mais
    équivalentes, d'écrire des expressions en mathématiques ou en
    programmation.

    Par exemple, l'opération d'addition de deux opérandes $x$ et $y$
    s'écrit en notation infixée
\begin{lstlisting}[backgroundcolor=\color{codebg}]
x + y
\end{lstlisting}
    En notation préfixée, aussi appelée notation polonaise,
    l'opérateur est placé \emph{avant} les opérandes:
\begin{lstlisting}[backgroundcolor=\color{codebg}]
+ x y
\end{lstlisting}
    On l'aura compris, en notation suffixée, ou notation polonaise
    inversée, l'opérateur apparait \emph{après} les opérandes:
\begin{lstlisting}[backgroundcolor=\color{codebg}]
x y +
\end{lstlisting}
    Nous sommes davantage habitués à lire la notation infixée, quoique
    la notation préfixée nous soit familière pour les opérateurs à un
    seul opérande (comme la négation) ou pour les appels de fonctions.
    La notation suffixée n'a jamais recours aux parenthèses.

    La compagnie HP commercialise de très prisées calculatrices
    scientifiques utilisant la notation suffixée (libellées RPN pour
    \emph{Reverse Polish Notation}) depuis 1972.
  \end{emphbox}
\end{figure}

\subsection{COBOL}
\label{sec:informatique:historique:cobol}

Complétons notre tour d'horizon des langages développés dans les
années 1950 et toujours en usage de nos jours par COBOL (\emph{COmmon
  Business Oriented Language}). Ce langage spécialisé dans les
applications de gestion a été créé en 1959 par un comité formé pour
proposer un langage commun pour l'administration américaine.
\begin{itemize}
\item Le COBOL reste très utilisé dans de grandes entreprises,
  notamment dans les institutions financières.
\item La légende urbaine veut que les programmeurs COBOL soient
  comparativement très bien rémunérés aujourd'hui sous l'effet combiné
  de leur rareté et de l'importance opérationnelle des applications
  qu'ils doivent maintenir.
\end{itemize}

\subsection{Algol}
\label{sec:informatique:historique:algol}

Dès la fin de la décennie 1950, un comité de scientifiques se réunit à
Zurich pour concevoir ce que l'on voudrait voir devenir le langage de
programmation standard. De ces rencontres naîtra Algol
(\emph{ALGorithmic Oriented Language}) en 1958.
\begin{itemize}
\item Comme la plupart des tentatives de définition d'un standard,
  c'est un échec: le langage est populaire dans les milieux
  académiques, mais restera peu utilisé dans les applications
  commerciales.
\item Néanmoins, on doit à Algol plusieurs innovations importantes, de
  telle sorte qu'un grand nombre des langages qui verront le jour par
  la suite seront considérés comme ses descendants. Le poster
  \link{http://www.oreilly.com/go/languageposter}{\emph{History of
      Programming Languages}} de O'Reilly Media illustre ce fait à
  merveille.
\item \citet{Hoare:1973} a d'ailleurs cette jolie formule: «Voici un
  langage très en avance sur son temps, il n'a pas seulement été une
  amélioration de ses prédécesseurs mais aussi une amélioration de
  presque tous ses successeurs.»
\end{itemize}

\begin{figure}[t]
  \centering
  \begin{minipage}{0.9\linewidth}
    \setkeys{Gin}{width=\textwidth}
    \includegraphics{standards} \\
    \footnotesize\sffamily%
    Tiré de \href{http://xkcd.com/927/}{XKCD.com}
  \end{minipage}
\end{figure}

\subsection{APL}
\label{sec:informatique:historique:apl}

Notre historique ne serait pas complet sans un mot sur APL (\emph{A
  Programming Language}, qui l'aurait cru). Même s'il n'a jamais connu
une diffusion importante, ce langage conçu par Kenneth Iverson autour
de 1962 n'en a pas moins eu une influence considérable sur la manière
de penser et de représenter les opérations mathématiques sur les
tableaux à plusieurs dimensions.

\begin{itemize}
\item Doté d'une large gamme de symboles pour représenter des
  opérations et d'une syntaxe tout à fait particulière --- les
  expressions sont exécutées de droite à gauche! --- l'APL est
  remarquablement concis et puissant; voir la
  \autoref{fig:informatique:apl} pour un aperçu.
\item Le revers de la cette médaille et ce qui a assurément nui à son
  adoption à large échelle, c'est la difficulté que l'on éprouve à
  relire le code. Assez pour que d'aucuns qualifient l'APL de «langage
  à écriture seulement».
\item APL a pendant longtemps été un langage très prisé par les
  actuaires, aussi subsiste-t-il du code dans certaines compagnies
  d'assurance.
\item Le modèle de traitement des vecteurs, matrices et tableaux de
  l'APL a servi d'inspiration pour la conception du langage S --- nous
  y reviendrons au \autoref{chap:presentation}.
\item Le langage continue sa vie aujourd'hui principalemnt sous forme
  de son successeur, J.
\end{itemize}

\begin{figure}
  \centering
  \scalebox{0.4}{\includegraphics{APL_intro}}
  \caption[Opérations simples en APL.]{Opérations simples en APL. De
    haut en bas: génération des nombres de $1$ à $5$ et stockage dans
    la variable \code{x}; affichage du contenu de \code{x}; somme des
    éléments de \code{x}; moyenne des éléments de \code{x}. {\small Source:
    François-Dominique,
    \link{https://commons.wikimedia.org/w/index.php?curid=43207460}{Wikimedia
      Commons}, CC BY-SA 4.0}}
  \label{fig:informatique:apl}
\end{figure}

\subsection{C}
\label{sec:informatique:historique:c}

Le dernier jalon important auquel nous nous attarderons est le langage
C. Celui-ci a été inventé en 1972 chez Bell Labs par Ken Thompson et
Dennis Ritchie afin de réécrire le système d'exploitation UNIX
(\autoref{sec:informatique:os:unix}). Il demeure beaucoup utilisé pour
la programmation système: le noyau de systèmes d'exploitation comme
Windows et Linux sont développés en grande partie en C.
\begin{itemize}
\item Le C est un langage de programmation généraliste considéré,
  selon les standards actuels, comme de bas niveau. Pour illustrer,
  l'utilisateur doit programmer des traitements comme la libération de
  la mémoire, la vérification de la validité des indices sur les
  tableaux, l'ouverture et la fermeture des fichiers, etc.
\item Il demeure un des langages les plus utilisés dans le monde et
  son influence est considérable. De nombreux langages plus modernes
  comme C$++$, C\# et Java reprennent des aspects de C.
\item Le C est également beaucoup utilisé pour le calcul numérique
  intensif, où il s'est en quelque sorte substitué à Fortran. La
  plupart des progiciels scientifiques --- dont R, encore une fois ---
  offrent la possibilité d'appeler du code C lorsque la rapidité de
  calcul devient un enjeu.
\end{itemize}

\begin{figure}[t]
  \notebox{Dans l'ouvrage désormais classique de \cite{KandR:1978}, le
    premier exemple d'un programme C affiche le message «\emph{hello,
      world}» à l'écran. Ça deviendra ensuite une tradition de
    démontrer le fonctionnement ou la syntaxe d'un langage avec cet
    exemple.}
\end{figure}

Python (1991)
\url{https://en.wikipedia.org/wiki/Python_(programming_language)}






\section{Syntaxe et sémantique}
\label{sec:informatique:syntaxe}

\section{Paradigmes de programmation}
\label{sec:informatique:paradigmes}

\section{Systèmes d'exploitation}
\label{sec:informatique:os}



%%% Local Variables:
%%% mode: latex
%%% TeX-engine: xetex
%%% TeX-master: "programmer-avec-r"
%%% End:
