%%% Copyright (C) 2018 Vincent Goulet
%%%
%%% Ce fichier fait partie du projet
%%% «Programmer avec R»
%%% https://gitlab.com/vigou3/programmer-avec-r
%%%
%%% Cette création est mise à disposition selon le contrat
%%% Attribution-Partage dans les mêmes conditions 4.0
%%% International de Creative Commons.
%%% https://creativecommons.org/licenses/by-sa/4.0/

\chapter*{Introduction}
\addcontentsline{toc}{chapter}{Introduction}
\markboth{Introduction}{Introduction}

Cet ouvrage traite de programmation informatique --- ou de codage,
pour utiliser le terme de plus en plus courant. À l'aube d'une
révolution numérique et de
l'\link{https://fr.wikipedia.org/wiki/Industrie_4.0}{indutrie 4.0}, la
programmation devient une compétence essentielle chez les étudiants
comme chez les travailleurs du vingt-et-unième siècle. De même, la
place sans cesse grandissante accordée à la science des données ---
analyse prédictive, apprentissage profond, intelligence artificielle
--- dans la plupart des disciplines scientifiques ne fait qu'accentuer
l'importance de disposer d'une solide formation en programmation.

Les principaux buts du présent ouvrage sont donc: d'abord de
développer une culture de l'informatique; ensuite d'acquérir la
capacité à résoudre des problèmes concrets à l'aide de l'algorithmique
et de la programmation; enfin de se familiariser avec les bonnes
pratiques reconnues en contexte de travail collaboratif, puisque la
programmation se pratique beaucoup plus souvent en groupe que seul.

Vous apprendrez à programmer en R, le langage au coeur de
l'environnement statistique du même nom et l'un des outils les plus
utilisés dans le monde pour l'analyse de données. Le système R connaît
depuis plus d'une décennie une progression remarquable dans ses
fonctionnalités, dans la variété de ses domaines d'application ou,
plus simplement, dans le nombre de ses utilisateurs. La documentation
sur l'utilisation de R en sciences naturelles, en sciences sociales,
en finance, etc, ne manque pas. Cependant, peu d'ouvrages se
concentrent sur l'apprentissage de R en tant que langage de
programmation sous-jacent aux fonctionnalités statistiques. C'est la
niche que je tâche d'occuper.

Outre l'algorithmique et la programmation avec R, l'ouvrage aborde des
outils informatiques qui vous permettront de gagner en efficacité et
en productivité, comme le système de gestion de version Git, les
expressions régulières et divers utilitaires d'analyse et de contrôle
de texte.

L'ouvrage vise d'abord à vous exposer à un maximum de code, puis à
vous faire pratiquer. C'est pourquoi plusieurs chapitres sont rédigés
de manière synthétique et qu'ils comportent peu d'exemples au fil du
texte. En revanche, vous serez appelé à lire et à exécuter le code
informatique se trouvant dans les sections d'exemples à la fin de
chacun des chapitres. Ce code, qui est distribué en version
électronique avec ce document, et les commentaires qui l'accompagnent
reviennent sur l'essentiel des concepts du chapitre et les
complémentent souvent. Je considère l'exercice d'«étude active»
consistant à exécuter du code et à voir ses effet comme essentiel à
l'apprentissage du langage R. Vous pouvez aussi inverser la
proposition et étudier le code informatique avant le texte du chapitre
correspondant si ce mode d'apprentissage vous convient mieux.




Pour télécharger le code R des sections d'exemples directement depuis
le \href{https://vigou3.github.io/introduction-programmation-r/}{site
  du projet} dans GitHub, utiliser le lien fourni à la page
précédente.

Un symbole de lecture vidéo dans la marge (comme ci-contre) indique
qu'une capsule vidéo est disponible dans la %
\capsule{http://www.youtube.com/user/VincentGouletIntroR/videos}{chaîne
  YouTube} %
de l'auteur sur le sujet en hyperlien.

Certains exemples et exercices trahissent le premier public de ce
document: on y fait à l'occasion référence à des concepts de base de
la théorie des probabilités et des mathématiques financières. Les
contextes actuariels demeurent néanmoins peu nombreux et ne devraient
généralement pas dérouter le lecteur pour qui ces notions sont moins
familières. Les réponses de tous les exercices se trouvent en annexe.
En consultation électronique, le numéro d'un exercice est un hyperlien
vers sa réponse, et vice versa.

On trouvera également en annexe de brèves introductions à l'éditeur de
texte GNU~Emacs et à l'environnement de développement intégré RStudio,
un court exposé sur la planification d'une simulation en R, ainsi que
des conseils sur l'administration d'une bibliothèque de packages R.

Nous tenons à remercier M.~Mathieu Boudreault pour sa collaboration
dans la rédaction des exercices et Mme~Mireille Côté pour la révision
linguistique de la seconde édition.

%%% Local Variables:
%%% mode: latex
%%% TeX-engine: xetex
%%% TeX-master: "programmer-avec-r"
%%% coding: utf-8
%%% End:
