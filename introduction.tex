\chapter*{Introduction}
\addcontentsline{toc}{chapter}{Introduction}
\markboth{Introduction}{Introduction}

Le système R connaît depuis plus d'une décennie une progression
remarquable dans ses fonctionnalités, dans la variété de ses domaines
d'application ou, plus simplement, dans le nombre de ses utilisateurs.
La documentation disponible suit la même tangente: plusieurs maisons
d'édition proposent dans leur catalogue des ouvrages --- voire des
collections complètes --- dédiés spécifiquement aux utilisations que
l'on fait de R en sciences naturelles, en sciences sociales, en
finance, etc. Néanmoins, peu d'ouvrages se concentrent sur
l'apprentissage de R en tant que langage de programmation sous-jacent
aux fonctionnalités statistiques. C'est la niche que nous tâchons
d'occuper.

Le présent ouvrage se base sur des notes de cours et des exercices
utilisés à l'École d'actuariat de l'Université Laval. L'enseignement
du langage R est axé sur l'exposition à un maximum de code --- que
nous avons la prétention de croire bien écrit --- et sur la pratique
de la programmation. C'est pourquoi les chapitres sont rédigés de
manière synthétique et qu'ils comportent peu d'exemples au fil du
texte. En revanche, le lecteur est appelé à lire et à exécuter le code
informatique se trouvant dans les sections d'exemples à la fin de
chacun des chapitres. Ce code et les commentaires qui l'accompagnent
reviennent sur l'essentiel des concepts du chapitre et les
complémentent souvent. Nous considérons l'exercice d'«étude active»
consistant à exécuter du code et à voir ses effet comme essentielle à
l'apprentissage du langage R.

Afin d'ancrer la présentation dans un contexte concret, plusieurs
chapitres proposent également d'entrée de jeu un problème à résoudre.
Nous fournissons des indications en cours de chapitre et la solution
complète à la fin. Afin d'être facilement identifiables, ces éléments
de contenu se présentent dans des encadrés de couleur contrastante et
marqués des symboles {\faCogs}, {\faBolt} et {\faLightbulbO}.

Le texte des sections d'exemples est disponible en format électronique
sous la %
\href{http://cran.r-project.org/other-docs.html}{rubrique de la documentation par des tiers} %
\emph{(Contributed)} du site \emph{Comprehensive R Archive Network}.
On peut obtenir directement l'archive en suivant le lien fournis à la
page précédente.

Un symbole de lecture vidéo dans la marge (comme ci-contre) indique
qu'une capsule vidéo est disponible dans la %
\capsule{http://www.youtube.com/user/VincentGouletIntroR/videos}{chaîne
  YouTube} %
de l'auteur sur le sujet en hyperlien.

Certains exemples et exercices trahissent le premier public de ce
document: on y fait à l'occasion référence à des concepts de base de
la théorie des probabilités et des mathématiques financières. Les
contextes actuariels demeurent néanmoins peu nombreux et ne devraient
généralement pas dérouter le lecteur pour qui ces notions sont moins
familières. Les réponses de tous les exercices se trouvent en annexe.
En consultation électronique, le numéro d'un exercice est un hyperlien
vers sa réponse, et vice versa.

On trouvera également en annexe de brèves introductions à l'éditeur de
texte GNU~Emacs et à l'environnement de développement intégré RStudio,
un court exposé sur la planification d'une simulation en R, ainsi que
des conseils sur l'administration d'une bibliothèque de packages R.

Nous tenons à remercier M.~Mathieu Boudreault pour sa collaboration
dans la rédaction des exercices et Mme~Mireille Côté pour la révision
linguistique de la seconde édition.

\begin{flushright}
  Vincent Goulet \\
  Québec, avril \year
\end{flushright}


%%% Local Variables:
%%% mode: latex
%%% TeX-master: "introduction_programmation_r"
%%% coding: utf-8
%%% End:
