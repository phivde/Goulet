%%% Copyright (C) 2017-2023 Vincent Goulet
%%%
%%% Ce fichier fait partie du projet
%%% «Programmer avec R»
%%% https://gitlab.com/vigou3/programmer-avec-r
%%%
%%% Cette création est mise à disposition sous licence
%%% Attribution-Partage dans les mêmes conditions 4.0
%%% International de Creative Commons.
%%% https://creativecommons.org/licenses/by-sa/4.0/

\chapter{Introduction}
\markboth{Introduction}{Introduction}

Cet ouvrage traite de programmation informatique. Au cœur de
l'actuelle révolution numérique, la programmation devient une
compétence essentielle chez les étudiants comme chez les travailleurs.
De même, la place sans cesse grandissante accordée à la science des
données --- analyse prédictive, apprentissage profond, intelligence
artificielle --- dans la plupart des disciplines scientifiques ne fait
qu'accentuer l'importance de disposer d'une solide formation en
programmation.

Les principaux buts du présent ouvrage sont donc: d'abord de
développer une culture de l'informatique; ensuite d'acquérir la
capacité à résoudre des problèmes concrets à l'aide de l'algorithmique
et de la programmation; enfin de se familiariser avec les bonnes
pratiques reconnues en contexte de travail collaboratif. En effet, la
programmation se pratique beaucoup plus souvent en groupe que seul.

Vous apprendrez à programmer en R, le langage au coeur de
l'environnement statistique du même nom et l'un des outils les plus
utilisés dans le monde pour l'analyse de données. Le système R connait
depuis plus d'une décennie une progression remarquable dans ses
fonctionnalités, dans la variété de ses domaines d'application ou,
plus simplement, dans le nombre de ses utilisateurs. La documentation
sur l'utilisation de R en sciences naturelles, en sciences sociales,
en finance, etc., ne manque pas. Cependant, peu d'ouvrages se
concentrent sur l'apprentissage du langage de programmation
sous-jacent aux fonctionnalités statistiques. C'est la niche que je
tâche d'occuper avec «Programmer avec R».

L'analyse de données requiert souvent de manipuler ou de traiter du
texte. Par conséquent, l'ouvrage aborde également les expressions
régulières et divers utilitaires d'analyse et de contrôle de texte.

Enfin, vous trouverez en annexe de brèves introductions à
l'environnement de développement intégré RStudio et à l'éditeur de
texte pour programmeurs GNU~Emacs, de même que les solutions complètes
des exercices.

\section*{Formations concomitantes}

Certaines parties de l'ouvrage tablent sur des connaissances de base
dans l'utilisation d'une ligne de commande Unix et d'un système de
gestion de versions. Mes formations concomitantes «Ligne de commande
Unix» \citep{Goulet:laboratoire-cli:2022} et «Gestion de versions avec
Git» \citep{Goulet:laboratoire-git:2023} permettent d'acquérir ces
connaissances.

La ligne de commande est la plus ancienne des interfaces avec les
ordinateurs. Si les interfaces graphiques ont à plusieurs égards
grandement facilité l'interaction homme-machine, elles n'ont pas pour
autant fait disparaitre ou rendu obsolète la ligne de commande, tout
particulièrement dans la pratique de la programmation.

Outil essentiel en contexte de travail collaboratif, le système de
gestion de versions facilite le suivi et le partage de fichiers, ainsi
que la mise en commun des contributions. Développé par Linus Torvalds
pour administrer le code source du noyau du système d'exploitation
Linux, \index{Git}Git est aujourd'hui le système de gestion de
versions le plus utilisé dans le monde.


\section*{Utilisation de l'ouvrage}

L'ouvrage vise d'abord à vous exposer à un maximum de code, puis à
vous faire pratiquer. C'est pourquoi plusieurs chapitres sont rédigés
de manière synthétique et qu'ils comportent peu d'exemples au fil du
texte.

En revanche, vous devrez lire et évaluer le code informatique se
trouvant dans les sections d'exemples à la fin de la plupart des
chapitres. Ce code et les commentaires qui l'accompagnent reviennent
sur l'essentiel des concepts du chapitre et les complémentent souvent.
Je considère l'exercice d'«étude active» consistant à évaluer du code
et à voir ses effets comme essentiel à l'apprentissage de la
programmation. Vous pouvez aussi inverser la proposition et étudier le
code informatique avant le texte du chapitre correspondant si ce mode
d'apprentissage vous convient mieux.

Le code des sections d'exemples est distribué avec le document sous
forme de fichiers de script. De plus, à chaque fichier \code{.R}
correspond un fichier \code{.Rout} contenant les résultats de son
évaluation non interactive.


\section*{Fonctionnalités interactives}

En consultation électronique, ce document se trouve enrichi de
plusieurs fonctionnalités interactives.
\begin{itemize}
\item Intraliens du texte vers une ligne précise d'une section de code
  informatique et, en sens inverse, du numéro de la ligne vers le
  point de la référence dans le texte. Ces intraliens sont marqués par
  la couleur \textcolor{link}{\rule{1.5em}{1.2ex}}.
\item Intraliens entre le numéro d'un exercice et sa solution, et vice
  versa. Ces intraliens sont aussi marqués par la couleur
  \textcolor{link}{\rule{1.5em}{1.2ex}}.
\item Intraliens entre les citations dans le texte et leur entrée dans
  la bibliographie. Ces intraliens sont marqués par la couleur
  \textcolor{citation}{\rule{1.5em}{1.2ex}}.
\item Hyperliens vers des ressources externes marqués par le symbole
  {\smaller\faExternalLink*} et la couleur
  \textcolor{url}{\rule{1.5em}{1.2ex}}.
\item Table des matières, liste des tableaux, liste des figures et
  liste des vidéos permettant d'accéder rapidement à des ressources du
  document.
\item Index comprenant, entre autres, les principaux mots clés du
  langage R et toutes leurs occurrences dans le texte ainsi que
  dans le code informatique.
\end{itemize}

\section*{Blocs signalétiques}

Le document est parsemé de divers types de blocs signalétiques
inspirés de
\link{https://asciidoctor.org/docs/user-manual/\#admonition}{AsciiDoc}
qui visent à attirer votre attention sur une notion.

\tipbox{Astuce! Ces blocs contiennent un truc, une astuce, ou tout
  autre type d'information utile.}
\vspace{-\baselineskip}

\warningbox{Avertissement! Ces blocs mettent l'accent sur une notion
  ou fournissent une information importante pour la suite.}
\vspace{-\baselineskip}

\cautionbox{Attention! Vous risquez de vous bruler ---
  métaphoriquement, s'entend --- si vous ne suivez pas les
  recommandations de ces blocs.}
\vspace{-\baselineskip}

\importantbox{Important! Ces blocs contiennent les remarques les plus
  importantes. Veillez à en tenir compte.}
\vspace{-\baselineskip}

\notebox{Fait amusant! Ces blocs contiennent des informations
  amusantes, mais non essentielles pour la compréhension de
  l'ouvrage.}
\vspace{-\baselineskip}

%% Utilisation de la commande \awesomebox pour illustrer les blocs de
%% vidéos pour éviter de créer une entrée dans la liste des vidéos.
\awesomebox{\aweboxrulewidth}{\faYoutube}{url}{Ces blocs
  contiennent des liens vers des vidéos dans ma %
  \link{https://www.youtube.com/user/VincentGouletIntroR/videos}{chaine
    YouTube}. Les vidéos sont répertoriées dans la liste des vidéos.}
\vspace{-\baselineskip}

\gotorbox{Ces blocs vous invitent à interrompre la lecture du texte
  pour passer à l'étude du code R des sections d'exemples.}
\vspace{-\baselineskip}

\macosbox{Remarques spécifiques à macOS.}
\vspace{-\baselineskip}

\windowsbox{Remarques spécifiques à Windows.}

\section*{Document libre}

Tout comme R et l'ensemble des outils présentés dans ce document, le
projet «Programmer avec R» s'inscrit dans le mouvement de
l'\link{https://www.gnu.org/philosophy/free-sw.html}{informatique
  libre}. Vous pouvez accéder à l'ensemble du code source en format
{\LaTeX} en suivant le lien dans la page de copyright. Vous trouverez
dans le fichier \code{README.md} toutes les informations utiles pour
composer le document.

Votre contribution à l'amélioration du document est également la
bienvenue; consultez le fichier \code{CONTRIBUTING.md} fourni avec ce
document et voyez votre nom ajouté au fichier \code{COLLABORATEURS}.

Bonne lecture!

%%% Local Variables:
%%% mode: latex
%%% TeX-engine: xetex
%%% TeX-master: "programmer-avec-r"
%%% coding: utf-8
%%% End:
