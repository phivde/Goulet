\chapter*{Introduction}
\addcontentsline{toc}{chapter}{Introduction}
\markboth{Introduction}{Introduction}


Depuis maintenant plus d'une décennie, le système R connaît une
remarquable progression dans ses fonctionnalités, dans la variété de
ses domaines d'application ou, plus simplement, dans le nombre de ses
utilisateurs. La documentation disponible a suivi la même tangente,
plusieurs maisons d'édition ayant d'ailleurs démarré des collections
dédiées spécifiquement aux utilisations que l'on fait de R en sciences
naturelles, en sciences sociales, en finance, etc. Néanmoins, peu
d'ouvrages se concentrent sur l'apprentissage de R en tant que langage
de programmation sous-jacent aux fonctions statistiques. C'est la
niche que nous tâchons d'occuper.

Cette troisième édition de l'ouvrage se distingue de la précédente à
plusieurs égards. Tout d'abord, le titre indique déjà que nous ne
traitons plus du système S-PLUS. S'il y a eu lutte d'importance et
d'influence entre le logiciel libre R et le logiciel commercial
S-PLUS, la cause est aujourd'hui entendue: R a gagné. Ensuite,
l'ensemble du texte a fait l'objet d'une révision et d'une mise à jour
en profondeur, en particulier les trois premiers chapitres. Enfin,
même si nous conservons un biais en faveur du tandem GNU~Emacs et ESS
pour interagir avec R, la présentation est moins liée à ces outils.

L'ouvrage est basé sur des notes de cours et des exercices utilisés à
l'École d'actuariat de l'Université Laval. L'enseignement du langage R
y est axé sur l'exposition à un maximum de code (que nous avons la
prétention de croire bien écrit) et sur la pratique de la
programmation. C'est pourquoi les chapitres sont rédigés de manière
synthétique et qu'ils comportent peu d'exemples au fil du texte. En
revanche, le lecteur est appelé à lire et à exécuter le code
informatique se trouvant dans les sections d'exemples à la fin de
chacun des chapitres. Ce code et les commentaires qui l'accompagnent
reviennent sur l'essentiel des concepts du chapitre et les
complémentent souvent. Nous considérons l'exercice d'«étude active»
comme essentielle à l'apprentissage du langage R.

Le texte des sections d'exemples est disponible en format électronique
sous la rubrique de la documentation par des tiers
(\emph{Contributed}) du site \emph{Comprehensive R Archive Network}:
\begin{quote}
  \url{http://cran.r-project.org/other-docs.html}
\end{quote}

Certains exemples et exercices trahissent le premier public de ce
document: on y fait à l'occasion référence à des concepts de base de
la théorie des probabilités et des mathématiques financières. Les
contextes actuariels demeurent néanmoins peu nombreux et ne devraient
généralement pas dérouter le lecteur pour qui ces notions sont moins
familières.

Nous tenons à remercier M.~Mathieu Boudreault pour sa collaboration
dans la rédaction des exercices et Mme~Mireille Côté pour la révision
linguistique de la seconde édition.

\begin{flushright}
  Vincent Goulet \\
  Québec, avril 2012
\end{flushright}


%%% Local Variables:
%%% mode: latex
%%% TeX-master: "introduction_programmation_r"
%%% coding: utf-8
%%% End:
