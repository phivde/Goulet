\chapter*{Introduction}
\addcontentsline{toc}{chapter}{Introduction}
\markboth{Introduction}{Introduction}

Depuis maintenant plus d'une décennie, le système R connaît une
progression remarquable dans ses fonctionnalités, dans la variété de
ses domaines d'application ou, plus simplement, dans le nombre de ses
utilisateurs. La documentation disponible a suivi la même tangente,
plusieurs maisons d'édition ayant démarré des collections dédiées
spécifiquement aux utilisations que l'on fait de R en sciences
naturelles, en sciences sociales, en finance, etc. Néanmoins, peu
d'ouvrages se concentrent sur l'apprentissage de R en tant que langage
de programmation sous-jacent aux fonctions statistiques. C'est la
niche que nous tâchons d'occuper.

L'ouvrage est basé sur des notes de cours et des exercices utilisés à
l'École d'actuariat de l'Université Laval. L'enseignement du langage R
est axé sur l'exposition à un maximum de code --- que nous avons la
prétention de croire bien écrit --- et sur la pratique de la
programmation. C'est pourquoi les chapitres sont rédigés de manière
synthétique et qu'ils comportent peu d'exemples au fil du texte. En
revanche, le lecteur est appelé à lire et à exécuter le code
informatique se trouvant dans les sections d'exemples à la fin de
chacun des chapitres. Ce code et les commentaires qui l'accompagnent
reviennent sur l'essentiel des concepts du chapitre et les
complémentent souvent. Nous considérons l'exercice d'«étude active»
consistant à exécuter du code et à voir ses effet comme essentielle à
l'apprentissage du langage R.

Le texte des sections d'exemples est disponible en format électronique
sous la rubrique de la documentation par des tiers
\emph{(Contributed)} du site \emph{Comprehensive R Archive Network}:
\begin{quote}
  \url{http://cran.r-project.org/other-docs.html}
\end{quote}

Cette quatrième édition de l'ouvrage se distingue principalement de la
précédente par l'ajout de liens vers des capsules vidéo réalisées par
l'auteur qui reviennent sur certains sujets plus délicats. Ces
capsules sont disponibles dans la chaîne YouTube
\begin{quote}
  \url{http://www.youtube.com/user/VincentGouletIntroR}
\end{quote}
Un symbole \marginpar{\raisebox{-14pt}[0em][0em]{\video}} de lecture
vidéo dans la marge --- tel que ci-contre --- indique qu'une capsule
vidéo est offerte sur le {\color{url} sujet} identifié par un
hyperlien.

Certains exemples et exercices trahissent le premier public de ce
document: on y fait à l'occasion référence à des concepts de base de
la théorie des probabilités et des mathématiques financières. Les
contextes actuariels demeurent néanmoins peu nombreux et ne devraient
généralement pas dérouter le lecteur pour qui ces notions sont moins
familières. Les réponses de tous les exercices se trouvent en annexe.
En consultation électronique, le numéro d'un exercice est un hyperlien
vers sa réponse, et vice versa.

On trouvera également en annexe une introduction à l'éditeur de texte
GNU~Emacs et au mode ESS, un bref exposé sur la planification d'une
simulation en R, ainsi que des conseils  sur l'administration d'une
bibliothèque de packages R.

Nous tenons à remercier M.~Mathieu Boudreault pour sa collaboration
dans la rédaction des exercices et Mme~Mireille Côté pour la révision
linguistique de la seconde édition.

\begin{flushright}
  Vincent Goulet \\
  Québec, janvier \year
\end{flushright}


%%% Local Variables:
%%% mode: latex
%%% TeX-master: "introduction_programmation_r"
%%% coding: utf-8
%%% End:
