\chapter{Installation de packages dans R}
\label{packages}

Un package R est un ensemble cohérent de fonctions, de jeux de données
et de documentation permettant de compléter les fonctionnalités du
système ou d'en ajouter de nouvelles. Les packages sont normalement
installés depuis le site \emph{Comprehensive R Archive
  Network} (CRAN; \url{http://cran.r-project.org}).

Cette annexe explique comment configurer R pour faciliter
l'installation et l'administration de packages externes.

Les instructions ci-dessous sont centrées autour de la création d'une
bibliothèque personnelle où seront installés les packages R
téléchargés de CRAN. Il est fortement recommandé de créer une telle
bibliothèque. Cela permet d'éviter les problèmes d'accès en écriture
dans la bibliothèque principale et de conserver les packages intacts
lors des mises à jour de R. Nous montrons également comment spécifier
le site miroir de CRAN pour éviter d'avoir à le répéter à chaque
installation de package.
\begin{enumerate}
\item Identifier le dossier de départ de l'utilisateur. En cas
  d'incertitude, examiner la valeur de la variable d'environnement
  \code{HOME}\footnote{%
    Pour les utilisateurs de GNU~Emacs sous Windows, la variable est
    créée par l'assistant d'installation de Emacs lorsqu'elle n'existe
    pas déjà.}, %
  depuis R avec la commande
\begin{Schunk}
\begin{Sinput}
> Sys.getenv("HOME")
\end{Sinput}
\end{Schunk}
  ou, pour les utilisateurs de Emacs, directement depuis l'éditeur avec
\begin{verbatim}
M-x getenv RET HOME RET
\end{verbatim}
  Nous référerons à ce dossier par le symbole \verb=~=.
\item Créer un dossier qui servira de bibliothèque de packages
  personnelle. Dans la suite, nous utiliserons \verb=~/R/library=.
\item Dans un fichier nommé \verb=~/.Renviron= (donc situé dans le
  dossier de départ), enregistrer la ligne suivante:
\begin{verbatim}
R_LIBS_USER="~/R/library
\end{verbatim}
  Au besoin, remplacer le chemin \verb=~/R/library= par celui du
  dossier créé à l'étape précédente. Utiliser la barre oblique avant
  (\code{/}) dans le chemin pour séparer les dossiers.
\item Dans un fichier nommé \verb=~/.Rprofile=, enregistrer l'option
  suivante:
\begin{verbatim}
options(repos = "http://cran.ca.r-project.org")
\end{verbatim}
  Si désiré, remplacer la valeur de l'option \code{repos} par l'URL
  d'un autre site miroir de CRAN.

  Les utilisateurs de GNU~Emacs voudront ajouter une autre option. Le
  code a entrer dans le fichier \verb=~/.Rprofile= sera plutôt
\begin{verbatim}
options(repos = "http://cran.ca.r-project.org",
        menu.graphics = FALSE)
\end{verbatim}
\end{enumerate}
Consulter la rubriques d'aide de \code{Startup} pour les détails sur
la syntaxe et l'emplacement des fichiers de configuration, celles de
\code{library} et \code{.libPaths} pour la gestion des bibliothèques
et celle de \code{options} pour les différentes options reconnues par
R.

Après un redémarrage de R, la bibliothèque personnelle aura préséance
sur la bibliothèque principale et il ne sera plus nécessaire de
préciser le site miroir de CRAN lors de l'installation de packages.
Ainsi, la simple commande
\begin{Schunk}
\begin{Sinput}
> install.packages("actuar")
\end{Sinput}
\end{Schunk}
téléchargera le package de fonctions actuarielles \textbf{actuar}
depuis le miroir canadien de CRAN et l'installera dans le dossier
\verb=~/R/library=. Pour charger le package en mémoire, on fera
\begin{Schunk}
\begin{Sinput}
> library("actuar")
\end{Sinput}
\end{Schunk}

On peut arriver au même résultat sans utiliser les fichiers de
configuration \code{.Renviron} et \code{.Rprofile}. Il faut cependant
recourir aux arguments \code{lib} et \code{repos} de la fonction
\code{install.packages} et à l'argument \code{lib.loc} de la fonction
\code{library}. Consulter les rubriques d'aide de ces deux fonctions
pour de plus amples informations.

%%% Local Variables:
%%% mode: latex
%%% TeX-master: "introduction_programmation_r"
%%% coding: utf-8
%%% End:
