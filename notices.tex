\begingroup
\calccentering{\unitlength}
\begin{adjustwidth*}{\unitlength}{-\unitlength}
  \setlength{\parindent}{0pt}
  \setlength{\parskip}{\baselineskip}
  \small

  {\textcopyright} {\year} Vincent Goulet \\

  \includegraphics[height=7mm,keepaspectratio=true]{by-sa}\\%
  Cette création est mise à disposition selon le contrat
  \href{http://creativecommons.org/licenses/by-sa/4.0/deed.fr}{%
    Attribution-Partage dans les mêmes conditions 4.0 International} de
  Creative Commons. En vertu de ce contrat, vous êtes libre de:
  \begin{itemize}
  \item \textbf{partager} --- reproduire, distribuer et communiquer
    l'{\oe}uvre;
  \item \textbf{remixer} --- adapter l'{\oe}uvre;
  \item utiliser cette {\oe}uvre à des fins commerciales.
  \end{itemize}
  Selon les conditions suivantes:

  \begin{tabularx}{\linewidth}{@{}lX@{}}
    \raisebox{-9mm}[0mm][13mm]{%
      \includegraphics[height=11mm,keepaspectratio=true]{by}} &
    \textbf{Attribution} --- Vous devez créditer l'{\oe}uvre, intégrer
    un lien vers le contrat et indiquer si des modifications ont été
    effectuées à l'{\oe}uvre. Vous devez indiquer ces informations par
    tous les moyens possibles, mais vous ne pouvez suggérer que
    l'Offrant vous soutient ou soutient la façon dont vous avez utilisé
    son {\oe}uvre. \\
    \raisebox{-9mm}{\includegraphics[height=11mm,keepaspectratio=true]{sa}}
    & \textbf{Partage dans les mêmes conditions} --- Dans le cas où vous
    modifiez, transformez ou créez à partir du matériel composant
    l'{\oe}uvre originale, vous devez diffuser l'{\oe}uvre modifiée dans
    les même conditions, c'est à dire avec le même contrat avec lequel
    l'{\oe}uvre originale a été diffusée.
  \end{tabularx}

  \textbf{Code source} \\
  \viewsource{\ghurl}

  \begin{tabularx}{1.0\linewidth}{@{}Xl@{}}
    Code informatique des sections d'exemples & \href{https://github.com/vigou3/introduction-programmation-r/releases/download/edition-\ednum/introduction-programmation-r-exemples.zip}{\downloadbutton} \\
    \addlinespace[3pt]
    Code source du document & \href{https://github.com/vigou3/introduction-programmation-r/}{\browsebutton}
  \end{tabularx}

  \fussy
  % ISBN {\ISBN} \\
  % Dépôt légal -- Bibliothèque et Archives nationales du Québec, {\year} \\
  % Dépôt légal -- Bibliothèque et Archives Canada, {\year}

  \textbf{Couverture} \\
  Le hibou en couverture est un harfang des neiges \emph{(Bubo
    scandiacus)}, l'emblème aviaire du Québec. Ce choix relève
  également d'un clin d'{\oe}il à la couverture de
  \cite{Braun:Rprogramming:2007}.
\end{adjustwidth*}
\endgroup

%%% Local Variables:
%%% mode: latex
%%% TeX-engine: xetex
%%% TeX-master: "programmer-avec-r"
%%% coding: utf-8
%%% End:
