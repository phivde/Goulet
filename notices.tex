%%% Copyright (C) 2017-2023 Vincent Goulet
%%%
%%% Ce fichier fait partie du projet
%%% «Programmer avec R»
%%% https://gitlab.com/vigou3/programmer-avec-r
%%%
%%% Cette création est mise à disposition sous licence
%%% Attribution-Partage dans les mêmes conditions 4.0
%%% International de Creative Commons.
%%% https://creativecommons.org/licenses/by-sa/4.0/

\begingroup
\calccentering{\unitlength}
\begin{adjustwidth*}{\unitlength}{-\unitlength}
  \setlength{\parindent}{0pt}
  \setlength{\parskip}{\baselineskip}
  \small

  {\smaller\ccbysa} 2017-{\year} par {\theauthor}. «\thetitle» est mis à
  disposition sous licence
  \href{https://creativecommons.org/licenses/by-sa/4.0/deed.fr}{%
    Attribution-Partage dans les mêmes conditions 4.0 International}
  de Creative Commons. En vertu de cette licence, vous êtes autorisé
  à:
  \begin{itemize}
  \item \textbf{partager} --- copier, distribuer et communiquer le
    matériel par tous moyens et sous tous formats;
  \item \textbf{adapter} --- remixer, transformer et créer à partir du
    matériel pour toute utilisation, y compris commerciale.
  \end{itemize}
  L'Offrant ne peut retirer les autorisations concédées par la licence
  tant que vous appliquez les termes de cette licence.

  Selon les conditions suivantes: \par
  \begin{tabularx}{\linewidth}{@{}lX@{}}
    \raisebox{-22pt}{\fontsize{32}{32}\selectfont\faCreativeCommonsBy}
    & \textbf{Attribution} --- Vous devez créditer l'œuvre, intégrer
      un lien vers la licence et indiquer si des modifications ont été
      effectuées à l'œuvre. Vous devez indiquer ces informations par
      tous les moyens raisonnables, sans toutefois suggérer que
      l'Offrant vous soutient ou soutient la façon dont vous avez utilisé
      son œuvre. \\
    \raisebox{-22pt}{\fontsize{32}{32}\selectfont\faCreativeCommonsSa}
    & \textbf{Partage dans les mêmes conditions} --- Dans le cas où vous
      effectuez un remix, que vous transformez, ou créez à partir du
      matériel composant l'œuvre originale, vous devez diffuser l'œuvre modifiée dans
      les mêmes conditions, c'est-à-dire avec la même licence avec laquelle
      l'œuvre originale a été diffusée.
  \end{tabularx}

  \textbf{Code source} \\
  \viewsource{\reposurl}

  \textbf{Crédits} \\
  La citation de Donald E.~Knuth en exergue est une traduction
  libre d'une citation tirée de: D'Agostino, S., «The Computer Scientist
  Who Can’t Stop Telling Stories»,
  \href{https://www.quantamagazine.org/computer-scientist-donald-knuth-cant-stop-telling-stories-20200416/}{\emph{Quanta Magazine}}.

  Image de peintre utilisée aux figures
  \ref*{fig:algorithmes:peinture-iter} et
  \ref*{fig:algorithmes:peinture-recursive} créée par
  \href{https://www.freepik.com}{iconicbestiary/Freepik}.

  \textbf{Couverture} \\
  Groupe d'écureuils de terre du Cap (\emph{Xerus inauris}) sortant de
  leur terrier près de Solitaire, dans le désert du Namib, en Namibie.
  Crédit photo: {\textcopyright} Hans Hillewaert,
  \href{https://creativecommons.org/licenses/by-sa/3.0/deed.fr}{CC
    BY-SA~3.0}, via
  \href{https://commons.wikimedia.org/w/index.php?curid=2429056}{Wikimedia
    Commons}.
\end{adjustwidth*}
\endgroup

%%% Local Variables:
%%% mode: latex
%%% TeX-engine: xetex
%%% TeX-master: "programmer-avec-r"
%%% coding: utf-8
%%% End:
