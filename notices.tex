\begingroup
\calccentering{\unitlength}
\begin{adjustwidth*}{\unitlength}{-\unitlength}
  \setlength{\parindent}{0pt}
  \setlength{\parskip}{\baselineskip}

  {\textcopyright} 2012 Vincent Goulet \\

  \includegraphics[height=7mm,keepaspectratio=true]{by-sa}\\%
  Cette création est mise à disposition selon le contrat
  \href{http://creativecommons.org/licenses/by-sa/2.5/ca/}{%
    Paternité-Partage à l'identique 2.5 Canada} de Creative Commons.
  En vertu de ce contrat, vous êtes libre de:
  \begin{itemize}
  \item \textbf{partager} --- reproduire, distribuer et communiquer
    l'{\oe}uvre;
  \item \textbf{remixer} --- adapter l'{\oe}uvre;
  \item utiliser cette {\oe}uvre à des fins commerciales.
  \end{itemize}
  Selon les conditions suivantes:

  \begin{tabularx}{\linewidth}{@{}lX@{}}
    \raisebox{-9mm}[0mm][13mm]{%
      \includegraphics[height=11mm,keepaspectratio=true]{by}} &
    \textbf{Attribution} --- Vous devez attribuer l'{\oe}uvre de la
    manière indiquée par l'auteur de l'{\oe}uvre ou le titulaire des
    droits (mais pas d'une manière qui suggérerait qu'ils vous
    soutiennent ou
    approuvent votre utilisation de l'{\oe}uvre). \\
    \raisebox{-9mm}{\includegraphics[height=11mm,keepaspectratio=true]{sa}}
    & \textbf{Partage à l'identique} --- Si vous modifiez, transformez
    ou adaptez cette {\oe}uvre, vous n'avez le droit de distribuer votre
    création que sous une licence identique ou similaire à celle-ci.
  \end{tabularx}

  \textbf{Code source} \\
  Le code source {\LaTeX} et R de ce document est disponible dans le
  dépôt Subversion
  \begin{quote}
    \url{https://svn.fsg.ulaval.ca/svn-pub/vgoulet/documents/intro_r/}
  \end{quote}
  ou en communiquant directement avec l'auteur.

  ISBN \ISBN \\
  Dépôt légal -- Bibliothèque nationale du Québec, 2012 \\
  Dépôt légal -- Bibliothèque et Archives Canada, 2012

  \textbf{Couverture} \\
  Le hibou en couverture est un harfang des neiges \emph{(Bubo
    scandiacus)}, l'emblème aviaire du Québec. Ce choix relève
  également d'un clin d'{\oe}il à la couverture de
  \cite{Braun:Rprogramming:2007}.

  Crédits photo: David G.~Hemmings via Wikimedia Commons
\end{adjustwidth*}
\endgroup

%%% Local Variables:
%%% mode: latex
%%% TeX-master: "introduction_programmation_r"
%%% coding: utf-8
%%% End:
