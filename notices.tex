\begingroup
\calccentering{\unitlength}
\begin{adjustwidth*}{\unitlength}{-\unitlength}
  \setlength{\parindent}{0pt}
  \setlength{\parskip}{\baselineskip}
  \small

  {\textcopyright} {\year} Vincent Goulet \\

  \input{licence}

  \textbf{Code source} \\
  \begin{tabularx}{1.0\linewidth}{@{}Xl@{}}
    Code informatique des sections d'exemples & \href{https://github.com/vigou3/introduction-programmation-r/releases/download/edition-\ednum/introduction-programmation-r-exemples.zip}{\downloadbutton} \\
    \addlinespace[3pt]
    Code source du document & \href{https://github.com/vigou3/introduction-programmation-r/}{\browsebutton}
  \end{tabularx}

  \fussy
  ISBN {\ISBN} \\
  Dépôt légal -- Bibliothèque et Archives nationales du Québec, {\year} \\
  Dépôt légal -- Bibliothèque et Archives Canada, {\year}

  \textbf{Couverture} \\
  Le hibou en couverture est un harfang des neiges \emph{(Bubo
    scandiacus)}, l'emblème aviaire du Québec. Ce choix relève
  également d'un clin d'{\oe}il à la couverture de
  \cite{Braun:Rprogramming:2007}.
\end{adjustwidth*}
\endgroup

%%% Local Variables:
%%% mode: latex
%%% TeX-engine: xetex
%%% TeX-master: "introduction_programmation_r"
%%% coding: utf-8
%%% End:
