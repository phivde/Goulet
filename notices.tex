\begingroup
\calccentering{\unitlength}
\begin{adjustwidth*}{\unitlength}{-\unitlength}
  \setlength{\parindent}{0pt}
  \setlength{\parskip}{\baselineskip}

  {\textcopyright} {\year} Vincent Goulet \\

  \input{licence}

  \sloppy
  \textbf{Code source} \\
  Le code source de ce document est disponible dans le
  dépôt \url{https://svn.fsg.ulaval.ca/svn-pub/vgoulet/documents/intro_r/}
  ou en commu\-ni\-quant directement avec l'auteur.

  \fussy
  ISBN {\ISBN} \\
  Dépôt légal -- Bibliothèque et Archives nationales du Québec, {\year} \\
  Dépôt légal -- Bibliothèque et Archives Canada, {\year}

  \textbf{Couverture} \\
  Le hibou en couverture est un harfang des neiges \emph{(Bubo
    scandiacus)}, l'emblème aviaire du Québec. Ce choix relève
  également d'un clin d'{\oe}il à la couverture de
  \cite{Braun:Rprogramming:2007}.

  Crédits photo: David G.~Hemmings via Wikimedia Commons
\end{adjustwidth*}
\endgroup

%%% Local Variables:
%%% mode: latex
%%% TeX-master: "introduction_programmation_r"
%%% coding: utf-8
%%% End:
