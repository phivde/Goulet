%%% Copyright (C) 2017 Vincent Goulet
%%%
%%% Ce fichier fait partie du projet «Programmer avec R»
%%% http://github.com/vigou3/programmer-avec-r
%%%
%%% Cette création est mise à disposition selon le contrat
%%% Attribution-Partage dans les mêmes conditions 4.0
%%% International de Creative Commons.
%%% http://creativecommons.org/licenses/by-sa/4.0/

\chapter{Analyse et contrôle de texte}
\label{chap:texte}

\begin{objectifs}
\item Déterminer si une chaine de caractères correspond ou non à une
  expression régulière donnée.
\item Écrire une expression régulière correspondant à une chaine de
  caractères donnée.
\item Identifier les lignes d'un fichier de texte ayant des propriétés
  communes à l'aide de \code{grep}.
\item Rechercher et remplacer des chaines de caractères ayant des
  propriétés communes à l'aide de \code{sed}.
\item Effectuer les opérations précédentes sur des chaînes de
  caractères divisées en champs logiques avec \code{awk}.
\item Effectuer les opérations de recherche et de remplacement de
  texte à l'aide d'expressions régulières avec les divers outils de R.
\end{objectifs}


Peut-être vous êtes-vous déjà demandé comment s'effectue la validation
de certains champs comme le code postal ou l'adresse de courrier
électronique dans les formulaires électroniques. Sûrement pas en
vérifiant si l'entrée figure dans la liste des quelques 17,5 millions
de codes postaux possibles ou, pire, parmi les milliards d'adresse de
courriel que les \link{https://tools.ietf.org/html/rfc3696}{règles
  internationales} permettent de concevoir! Non, ce qu'il faut, c'est
un «langage» qui permet de décrire \emph{comment} une chaine de
caractères peut être composée, sans toutefois en fixer la composition
exacte. Un tel langage existe: ce sont les \emph{expressions
  régulières} ou expressions relationnelles.

Une expression régulière (\emph{regular expression}, ou regex ou
regexp) est une chaine de caractères qui décrit, selon une syntaxe
précise, un ensemble de chaines de caractères possibles
\citep{Wikipedia:Expression_reguliere}.

Par exemple, les plaques d'immatriculation récentes distribuées par la
Société d'assurance automobile du Québec (SAAQ) sont composées ainsi:
une lettre, deux chiffres, une espace, trois lettres. Les lettres
\textsf{O}, \textsf{I} et \textsf{U} ne sont pas utilisées à cause de
leur similitude avec, respectivement, les chiffres {\fullcaps 0} et
{\fullcaps 1} et la lettre \textsf{V}. L'expression régulière suivante
permet de décrire l'ensemble des numéros de plaques possibles (y
compris les combinaisons de lettres exclues comme \code{LSD} ou
\code{SEX}):
\begin{Schunk}
\begin{Verbatim}
[A-HJ-NP-TV-Z][0-9]{2} [A-HJ-NP-TV-Z]{3}
\end{Verbatim}
\end{Schunk}

Voici quelques autres utilisations possibles des expressions
régulières.
\begin{itemize}
\item Rechercher du texte pouvant contenir des variations ou des
  fautes d'orthographe comme, par exemple, «je transfert» plutôt que
  «je transfère», ou encore les multiples variations autour du verbe
  «appeler»: appel, appelle, appellent, appeler, appelez, etc.
\item Remplacer le nom d'une commande par un autre lorsque la commande
  s'étend de part et d'autre d'un argument qui, lui, change d'une fois
  à l'autre. Par exemple, dans le système de mise en page {\LaTeX}, la
  syntaxe des commandes est la suivante:
  \cmdprint{\cmd}\oarg{option}\marg{arg}. Pour remplacer toutes les
  utilisations d'une commande \cmdprint{\foo} par \cmdprint{\bar}, il
  faut pouvoir tenir compte de la présence possible d'un argument
  optionnel \oarg{option} ainsi que de l'argument \marg{arg} qui
  change d'une utilisation de la commande à une autre.
\item Extraire les coordonnées géographiques d'un lieu (latitude et
  longitude) de l'URL d'une carte Google Maps. Par exemple, l'%
  \link{https://www.google.ca/maps/place/Universite+Laval/@46.7817463,-71.2769311,17z/data=!3m1!4b1!4m5!3m4!1s0x4cb896c469ff32f9:0x15feb853bd2f8247!8m2!3d46.7817463!4d-71.2747424}{URL
    correspondant} à la recherche «Université Laval» est
  \begin{Schunk}
\begin{Verbatim}
https://www.google.ca/maps/place/Universite+Laval/
  @46.7817463,-71.2769311,17z/data=!3m1!4b1!4m5!3m4!
  1s0x4cb896c469ff32f9:0x15feb853bd2f8247!8m2!3d
  46.7817463!4d-71.2747424
\end{Verbatim}
  \end{Schunk}
  dont on décode que le lieu se trouve à une latitude de
  $\nombre{46,7817463}$ et à une longitude de $\nombre{-71.2769311}$
  en degrés décimaux.
\end{itemize}

Un jour ou l'autre, vous aurez à traiter des chaines de caractères en
programmation ou en analyse de données. Ce jour-là, connaitre les
expressions régulières vous sauvera énormément de temps.


\section{Outils d'analyse et de contrôle du texte}
\label{sec:texte:outils}

L'expression régulière ne constitue qu'un langage de description d'une
chaine de texte. Pour exploiter ce langage, des outils sont
nécessaires. Aux fins de cet ouvrage, j'ai choisi de concentrer notre
étude sur les utilitaires Unix standards en ligne de commande
\code{grep}, \code{sed} et \code{awk}. Ils font partie intégrante des
systèmes d'exploitation macOS et Linux et, sous Windows, ils sont
livrés avec les interpréteurs de commande MSYS de \index{MSYS2}MSYS2
et \index{Git~Bash}Git~Bash de \emph{Git for Windows} dont nous avons
déjà parlé à la \autoref{sec:informatique:cli}.

Comme la plupart des utilitaires Unix, \Icode{grep}, \Icode{sed} et
\Icode{awk} prennent en entrée un flux de texte ou un ou plusieurs
fichiers en format texte brut et ils opèrent sur cette entrée ligne
par ligne. Le résultat du traitement est affiché à la ligne de
commande, envoyé à un autre programme ou redirigé vers un fichier. Les
outils diffèrent quant au type de traitement qu'ils peuvent effectuer.
\begin{description}
\item[\code{grep}] sélectionne les lignes en entrée correspondant à
  une expression régulière;
\item[\code{sed}] sert surtout pour rechercher et remplacer du texte;
\item[\code{awk}] permet de traiter aisément du texte séparé en champs
  ainsi que du texte réparti sur plusieurs lignes.
\end{description}

\warningbox{Il existe plusieurs versions différentes de ces outils,
  parfois avec des différences importantes, notamment en ce qui a
  trait à \code{sed}. Les versions sous Linux, \index{MSYS2}MSYS2 et
  \index{Git~Bash}Git~Bash sont généralement celles du
  \link{https://www.gnu.org/}{projet GNU}. Dans macOS, elles
  proviennent de
  \link{https://en.wikipedia.org/wiki/Berkeley_Software_Distribution}{BSD}.}


\section{Transfert de données et redirection}
\label{sec:texte:flux}

Parce qu'ils sont souvent utilisés avec les outils \icode{grep},
\icode{sed} et \icode{awk}, il convient, avant d'aller plus loin, de
dire un mot sur les opérateurs de transfert de données et de
redirection de \index{Unix}Unix.

Les programmes Unix en ligne de commande reçoivent leurs données d'une
\Index{entree standard@entrée standard}\Index{Unix!entree
  standard@entrée standard}\emph{entrée standard} (\emph{standard
  input}, souvent abrégé en \emph{stdin}) et renvoient leurs résultats
vers la \Index{sortie standard}\Index{Unix!sortie
  standard}\emph{sortie standard} (\emph{standard output},
\emph{stdout}). Pour simplifier, l'entrée standard d'un terminal
serait le clavier et la sortie standard, l'écran.

L'opérateur de transfert de données %
\index{{"|}@\code{\textbar} (Bash)|bfhyperpage} «\code{\textbar}»
(nommé «tuyau», de l'anglais \emph{pipe}) permet de
transférer la sortie d'un programme directement à l'entrée d'un autre
programme. Autrement dit, plutôt que d'afficher ses résultats à
l'écran, le premier programme les passe en entrée au second programme.

L'opérateur de redirection %
\index{>@\code{>} (Bash)|bfhyperpage}«\code{>}» permet, lui, de
rediriger la sortie d'un programme vers un fichier afin de sauvegarder
les résultats. Si le fichier existe déjà, son contenu est écrasé pour
le nouveau contenu. Pour ajouter le nouveau contenu à la fin du
fichier, il faut utiliser la variante double %
\index{>>@\code{>>} (Bash)|bfhyperpage}«\code{>>}» de l'opérateur de
redirection.

Enfin, l'opérateur de redirection %
\Index{>@\code{<} (Bash)}«\code{<}» permet, à l'inverse du précédent,
de déverser le contenu d'un fichier dans l'entrée standard d'un
programme. La variante double existe également.

La documentation citée en référence à la section suivante utilise
abondamment les opérateurs ci-dessus.


\section{Expressions régulières}
\label{sec:texte:regex}

Il existe une multitude de tutoriels et de documents de référence sur
les expressions régulières et sur les utilitaires \icode{grep},
\icode{sed} et \icode{awk}, que ce soit dans Internet ou sous forme de
livre. Or, le guide \link{https://developer.apple.com/library/content/documentation/OpenSource/Conceptual/ShellScripting/}{\emph{Shell Scripting Primer}} de la bibliothèque
pour développeurs de Apple \citep{Apple:shellprimer} s'avère une
référence idéale pour vous accompagner dans l'atteinte des objectifs
d'apprentissage mentionnés au début du chapitre.

\windowsbox{Utilisateurs de Windows: pas de souci! Bien qu'émanant de
  Apple, le guide traite de procédures et d'outils Unix standards,
  outils dont vous disposez si vous avez installé \index{Git}\emph{Git
    for Windows} ou MSYS2\index{MSYS2}, tel qu'expliqué à la
  \autoref{sec:informatique:cli:windows}.}

Je vous invite donc à étudier les chapitres suivants du guide
\emph{Shell Scripting Primer}:
\begin{itemize}
\item
  \link{https://developer.apple.com/library/content/documentation/OpenSource/Conceptual/ShellScripting/RegularExpressionsUnfettered/RegularExpressionsUnfettered.html}{\emph{Regular
      Expressions Unfettered}} jusqu'à la section \emph{Using
    Modifiers}, inclusivement.
\item
  \link{https://developer.apple.com/library/content/documentation/OpenSource/Conceptual/ShellScripting/Howawk-ward/Howawk-ward.html}{\emph{How
      AWK-ward}} jusqu'à la section \emph{Changing the Record and
    Field Separators in AWK Scripts}, inclusivement.
\end{itemize}
Revenez ensuite ici pour des exemples additionnels et des exercices.

\warningbox{Vous pouvez ignorer toutes les notions relatives à Perl
  exposées dans \emph{Shell Scripting Primer}.}


\section{Exemples additionnels}
\label{sec:texte:exemples}

Exceptionnellement, je présente les exemples additionnels de ce
chapitre en texte normal plutôt que sous forme de fichier de script.

Les exemples ci-dessous utilisent des fichiers distribués avec le
présent document dans l'archive \code{programmer-avec-r.zip}.
Assurez-vous d'avoir extrait les fichiers de l'archive quelque part
sur votre disque et notez bien cet endroit: vous en aurez besoin dans
un instant.

Vous devez entrer les expressions ci-dessous à une ligne de commande
du \index{Terminal}Terminal sous macOS, ou de \index{Git~Bash}Git~Bash
ou MSYS sous Windows.

\importantbox{L'espace horizontal étant compté dans un document en
  format PDF, il est parfois nécessaire de scinder une commande sur
  deux lignes ou plus. Pour ces cas, j'ai inséré le symbole de
  continuation de ligne «\bs» de Bash à la fin des lignes. Vous pouvez
  ainsi copier les commandes directement du document et les coller
  dans une ligne de commande. Le symbole {\bs} ne fait \emph{pas}
  partie des commandes.}

En premier lieu, déplacez, à l'aide de la commande \icode{cd},
l'invite de commande dans le répertoire où vous avez extrait les
fichiers de l'archive.

Comme premier exemple, dressons la liste de toutes les utilisations de
la fonction \code{matrix} dans les fichiers d'exemples. C'est un
travail pour l'utilitaire \icode{grep}.
\begin{Schunk}
\begin{Verbatim}
$ grep matrix *.R
\end{Verbatim}
\end{Schunk}

L'option \code{-l} de \icode{grep} permet de limiter les résultats à
la liste des fichiers contenant au moins une utilisation de
\code{matrix}.
\begin{Schunk}
\begin{Verbatim}
$ grep -l matrix *.R
\end{Verbatim}
\end{Schunk}

La commande suivante permet de savoir quel fichier d'exemples traite
de la suite de Fibonacci.
\begin{Schunk}
\begin{Verbatim}
$ grep -l fibonacci *.R
\end{Verbatim}
\end{Schunk}

Pour connaitre les fichiers d'exemples dans lesquels apparaissent les
objets \code{letters} et \code{LETTERS}, nous devons rechercher sans
égard à la casse pour attraper les deux cas.
\begin{Schunk}
\begin{Verbatim}
$ grep -i letters *.R
\end{Verbatim}
\end{Schunk}

Allons-y maintenant d'une expression régulière plus élaborée pour
trouver toutes les utilisations de la fonction \code{g} dans les
fichiers d'exemples, c'est-à-dire la chaine \code{g(} précédée d'au
moins une espace ou en début de ligne.
\begin{Schunk}
\begin{Verbatim}
$ grep -E '( |^)g\(' *.R
\end{Verbatim}
\end{Schunk}

Tel que mentionné à la \autoref{sec:texte:outils}, la fonction
\icode{sed} sert surtout pour rechercher et remplacer dans un fichier,
et ce, ligne par ligne par défaut. Remplaçons les symboles de
commentaires triples \code{\#\#\#} en début de ligne (à la Emacs) dans
le fichier \code{implementation.R} par un symbole unique \code{\#} (à
la RStudio). Pour faire bonne mesure, plaçons le résultat dans un
nouveau fichier \code{implementation-new.R} à l'aide de l'opérateur de
redirection \index{>@\code{>} (Bash)}\code{>}.
\begin{Schunk}
\begin{Verbatim}
$ sed 's/^###/#/' implementation.R > implementation-new.R
\end{Verbatim}
\end{Schunk}

Le programme \icode{awk} est particulièrement utile pour traiter des
fichiers dont les lignes sont séparées en «champs», comme des colonnes
de données, par exemple. Or, le présent document est justement livré
avec le fichier \code{100metres.data} qui contient la liste des 31
meilleurs temps enregistrés au 100~mètres homme entre 1964 et 2005
(voir l'\autoref{ex:internes:100metres}). La commande \Icode{cat}
permet d'afficher le contenu du fichier sur la sortie standard.
\begin{Schunk}
\begin{Verbatim}
$ cat 100metres.data
\end{Verbatim}
\end{Schunk}

Effectuons dans un premier temps l'extraction des temps des records
(second champ). La première commande utilise un transfert de données
du programme \icode{cat} vers \icode{awk} avec l'opérateur %
\index{{"|}@\code{\textbar} (Bash)}\code{\textbar}, alors
que la seconde, plus usuelle et tout à fait équivalente, emploie un
fichier en entrée.
\begin{Schunk}
\begin{Verbatim}
$ cat 100metres.data | awk '{ print $2 }'
$ awk '{ print $2 }' 100metres.data
\end{Verbatim}
\end{Schunk}

L'inversion des deux colonnes du fichier est simple à réaliser avec
\icode{awk}.
\begin{Schunk}
\begin{Verbatim}
$ awk '{ print $2, $1 }' 100metres.data
\end{Verbatim}
\end{Schunk}

Les colonnes du fichier \code{100metres.data} sont séparées par une
espace. Remplaçons ces espaces par des virgules pour convertir le
fichier en format CSV. Je fournis deux solutions, avec \icode{awk} et avec
\icode{sed}.
\begin{Schunk}
\begin{Verbatim}
$ awk '{ print $1","$2 }' 100metres.data
$ sed 's/ /,/' 100metres.data
\end{Verbatim}
\end{Schunk}

Changeons maintenant le format de la date du format
\link{https://fr.wikipedia.org/wiki/ISO_8601}{ISO~8601} aaaa-mm-qq
vers le format américain qq/mm/aa. D'abord une solution avec
\icode{awk}. En premier lieu, nous devons changer le séparateur de
champs de \code{awk} (l'espace par défaut) pour l'espace et le tiret.
Ainsi, \code{awk} va non seulement séparer les deux colonnes du
fichier, mais aussi les champs de la date. Ensuite, nous replaçons les
champs dans l'ordre voulu avec les bons séparateurs. La fonction
\index{substr@\code{substr} (awk)}\code{substr} de \code{awk} permet
de sélectionner une partie d'une chaine de caractères
\citep[section~9.1.3]{awk}.
\begin{Schunk}
\begin{Verbatim}
$ awk 'BEGIN { FS = "[ -]" } \
       { print $2"/"$3"/"substr($1, 3), $4 }' \
       100metres.data
\end{Verbatim}
\end{Schunk}

La solution avec \icode{sed} maintenant, qui repose non pas sur des
champs détectés automatiquement, mais plutôt sur la recherche et le
remplacement d'expressions régulières. Nous recherchons l'expression
régulière suivante en capturant chaque élément trouvé (sauf les
tirets): les nombres 19 ou 20; deux chiffres; un tiret; deux chiffres;
un tiret; deux chiffres; tout le reste de la ligne. Nous replaçons
ensuite les éléments capturés dans l'ordre souhaité avec des barres
obliques \code{/} entre les éléments de dates. Comme le symbole
\code{/} est utilisé dans la chaine de sortie, il faut utiliser un
autre symbole pour séparer les champs de la commande \code{sed}. J'ai
utilisé le symbole \code{~} ici.
\begin{Schunk}
\begin{Verbatim}
$ sed -E 's~(19|20)([0-9]{2})-([0-9]{2})-([0-9]{2})(.*)~'\
'\3/\4/\2\5~' 100metres.data
\end{Verbatim}
\end{Schunk}

Amusons-nous maintenant avec le fichier \code{README.md} qui contient,
regroupées à la fin du fichier, les notes de mise à jour des différentes
versions du document. Les deux commandes suivantes dressent la liste
des numéros de versions avec les variantes de base et étendue
des expressions régulières.
\begin{Schunk}
\begin{Verbatim}
$ grep '^### [[:digit:]]' README.md
$ grep -E '^#{3} [[:digit:]]' README.md
\end{Verbatim}
\end{Schunk}

Terminons avec deux exemples plus avancés. Nous souhaitons d'abord
extraire tout l'historique des versions du fichier \code{README.md}.
Cet historique se trouve à la fin du fichier, à partir de la ligne
débutant par la chaine \code{\#\# Historique}. Le meilleur outil pour
effectuer des opérations sur plusieurs lignes d'un fichier est
\icode{awk}, tel que mentionné à la \autoref{sec:texte:outils}.
L'idée, ici, est la suivante: lorsque la chaine est trouvée, donner
une valeur de 1 (vrai) à une variable d'«état», puis afficher les
lignes lorsque la variable est vraie (donc toutes celles qui suivent).
Je présente deux versions de la commande, une explicite et une plus
compacte.
\begin{Schunk}
\begin{Verbatim}
$ awk '/^## Historique/ { state = 1; next } \
       state == 1 { print }' README.md
$ awk '/^## Historique/ { state = 1; next } \
       state' README.md
\end{Verbatim}
\end{Schunk}

\tipbox{Explication de la version compacte: écrire \code{state == 1}
  pour tester que la variable \code{state} est vraie n'est pas
  nécessaire dans la mesure où il est sous-entendu que l'expression
  qui précède l'action dans une commande \icode{awk} est un test
  conditionnel et que \code{awk} traite la valeur 1 comme vraie. De
  plus, la commande \code{print} est implicite avec \code{awk}, donc
  il n'est pas nécessaire de la demander explicitement.}

Ensuite, nous voulons extraire seulement les notes de mise à jour de
la \emph{dernière} version du document. L'idée consiste à afficher les lignes
du fichier à partir de la première section marquée par \code{\#\#\#} après
\code{\#\# Historique} et d'arrêter lorsqu'une autre section marquée par
\code{\#\#\#} est atteinte. Ces opérations nécessitent deux valeurs de la
variable d'état \code{state}.
\begin{Schunk}
\begin{Verbatim}
$ awk '/^## Historique/ { state = 1 } \
       (state == 1) && /^### / { state = 2; next } \
       (state == 2) && /^### / { exit } \
       state == 2 { print }' README.md
\end{Verbatim}
\end{Schunk}

\notebox{Les opérations de composition du présent document, de
  publication dans GitHub et de mise à jour de la page web sont
  entièrement automatisées à l'aide de
  \link{https://fr.wikipedia.org/wiki/Make}{\code{make}}, un autre de
  ces outils standards sur les systèmes Unix. Vous pouvez retrouver
  une variante un peu plus élaborée de la commande \code{awk}
  ci-dessus dans la règle \code{create-release} du fichier de
  configuration \code{Makefile} qui se trouve dans le
  \link{\ghurl}{code source} du document.}

\section{Exercices}
\label{sec:texte:exercices}

\Opensolutionfile{solutions}[solutions-texte]

\begin{Filesave}{solutions}
\section*{Chapitre \ref*{chap:texte}}
\addcontentsline{toc}{section}{Chapitre \protect\ref*{chap:texte}}

\end{Filesave}


\begin{exercice}
  Dans chacun des cas ci-dessous, déterminer à laquelle ou lesquelles
  des chaines de caractères correspond l'expression régulière donnée.
  Les symboles \verb=/= ne servent qu'à délimiter le début et la fin
  des expressions régulières. Ne pas tenir compte des espaces avant ou
  après les chaines.\footnote{%
    Exercice adapté de
    \url{https://regex.sketchengine.co.uk/extra_regexps.html}.}
  \begin{enumerate}
    \setlength{\multicolsep}{2pt}
  \item \verb~/a(ab)*a/~
    \begin{enumerate}[1)]
      \begin{multicols}{3}
      \item \verb~abababa~
      \item \verb~aaba~
      \item \verb~aabbaa~
      \end{multicols}
      \begin{multicols}{3}
      \item \verb~aba~
      \item \verb~aabababa~
      \end{multicols}
    \end{enumerate}
  \item \verb~/ab+c?/~
    \begin{enumerate}[1)]
      \begin{multicols}{3}
      \item \verb~abc~
      \item \verb~ac~
      \item \verb~abbb~
      \end{multicols}
      \begin{multicols}{3}
      \item \verb~bbc~
      \item \verb~abcc~
      \end{multicols}
    \end{enumerate}
  \item \verb~/a.[bc]+/~
    \begin{enumerate}[1)]
      \begin{multicols}{3}
      \item \verb~abc~
      \item \verb~abbbbbbbb~
      \item \verb~azc~
      \end{multicols}
      \begin{multicols}{3}
      \item \verb~abcbcbcbc~
      \item \verb~ac~
      \item \verb~asccbbbbcbcccc~
      \end{multicols}
    \end{enumerate}
  \item \verb~/abc|xyz/~
    \begin{enumerate}[1)]
      \begin{multicols}{3}
      \item \verb~abc~
      \item \verb~xyz~
      \item \verb~abc|xyz~
      \end{multicols}
    \end{enumerate}
  \item \verb~/[a-z]+[\.\?!]/~
    \begin{enumerate}[1)]
      \begin{multicols}{3}
      \item \verb~battle!~
      \item \verb~Hot~
      \item \verb~green~
      \end{multicols}
      \begin{multicols}{3}
      \item \verb~swamping.~
      \item \verb~jump up.~
      \item \verb~undulate?~
      \end{multicols}
      \begin{multicols}{3}
      \item \verb~is.?~
      \end{multicols}
    \end{enumerate}
  \item \verb~/[a-zA-Z]*[^,]=/~
    \begin{enumerate}[1)]
      \begin{multicols}{3}
      \item \verb~Butt=~
      \item \verb~BotHEr,=~
      \item \verb~Ample~
      \end{multicols}
      \begin{multicols}{3}
      \item \verb~FIdDlE7h=~
      \item \verb~Brittle =~
      \item \verb~Other.=~
      \end{multicols}
    \end{enumerate}
  \item \verb~/[a-z][\.\?!]\s+[A-Z]/~
    \begin{enumerate}[1)]
      \begin{multicols}{3}
      \item \verb~A. B~
      \item \verb~c! d~
      \item \verb~e f~
      \end{multicols}
      \begin{multicols}{3}
      \item \verb~g. H~
      \item \verb~i? J~
      \item \verb~k L~
      \end{multicols}
    \end{enumerate}
  \item \verb~/<[^>]+>/~
    \begin{enumerate}[1)]
      \begin{multicols}{2}
      \item \verb~<an xml tag>~
      \item \verb~<opentag> <closetag>~
      \end{multicols}
      \begin{multicols}{2}
      \item \verb~</closetag>~
      \item \verb~<>~
      \end{multicols}
      \begin{multicols}{2}
      \item \verb~<with attribute=”77”>~
      \end{multicols}
    \end{enumerate}
  \end{enumerate}

  \begin{sol}
    \begin{enumerate}
    \item L'expression régulière recherche «\verb=a=», suivi zéro ou
      plusieurs occurrences de «\verb=ab=», suivi de «\verb=b=». Les
      choix 2 et 5 satisfont ces conditions.
    \item L'expression régulière recherche «\verb=a=», suivi d'au
      moins une occurrence de «\verb=b=», suivi de zéro ou une
      occurrence de «\verb=c=». Les choix 1 et 3 satisfont ces
      conditions.
    \item L'expression régulière recherche «\verb=a=», suivi d'un
      caractère quelconque (incluant «\verb=b=» ou «\verb=c=»), suivi
      d'au moins une occurrence de «\verb=b=» ou de «\verb=c=». Les
      choix 1, 2, 3, 4 et 6 satisfont ces conditions.
    \item L'expression régulière recherche «\verb=abc=» ou
      «\verb=xyz=». Les choix 1 et 2 satisfont la condition. Le
      troisième choix n'est pas valide puisque l'expression ne
      recherche pas le symbole «\verb=|=».
    \item L'expression régulière recherche au moins lettre minuscule
      (à l'exclusion de tout autre symbole), suivi de l'un ou l'autre
      des symboles «\verb=.=», «\verb=?=», «\verb=!=» (sans
      répétition). Les choix 1, 4 et 6 satisfont ces conditions.
    \item L'expression régulière recherche zéro ou plusieurs lettres
      minuscules ou majuscule (mais aucun autre symbole), suivi de
      tout autre caractère qu'une virgule, suivi du symbole
      «\verb|=|». Les choix 1, 5 et 6 satisfont ces conditions.
    \item L'expression régulière recherche une lettre minuscule, suivi
      de l'un ou l'autre des symboles «\verb=.=», «\verb=?=»,
      «\verb=!=» (sans répétition), suivi d'au moins une espace, suivi
      d'une lettre majuscule. Les choix 4 et 5 satisfont ces
      conditions.
    \item L'expression régulière recherche un symbole «\verb=<=»,
      suivi d'au moins symbole autre que «\verb=>=», suivi du symbole
      «\verb=>=». Les choix 1, 3 et 5 satisfont ces conditions.
    \end{enumerate}
  \end{sol}
\end{exercice}

\begin{exercice}
  Composer une expression régulière qui correspond à tous les mots de
  la liste de gauche, ci-dessous, mais à aucun des mots de la liste de
  droite.\footnote{%
    Exercice adapté de
    \url{https://regex.sketchengine.co.uk/cgi/ex1.cgi}.}
  \begin{center}
    \begin{minipage}[t]{0.3\linewidth}
      pit \\
      spot \\
      spate \\
      slap two \\
      respite
    \end{minipage}
    \begin{minipage}[t]{0.3\linewidth}
      pt \\
      Pot \\
      peat \\
      part
    \end{minipage}
  \end{center}
  \begin{sol}
    L'expression doit correspondre à la lettre minuscule «\verb=p=»
    précédée ou non d'une ou plusieurs lettres (ou symboles, ce n'est
    pas spécifié), suivie d'une (et une seule) lettre ou d'une espace,
    de la lettre «\verb=t=» et, enfin, de zéro ou de plusieurs
    lettres. Il y a assurément plusieurs réponses valides. En voici
    une: \verb=/.*p[a-z ]t.*/=.
  \end{sol}

\end{exercice}

\begin{exercice}
  Composer une expression régulière qui permet de vérifier la validité
  d'un code postal canadien dans un formulaire électronique. Ne pas
  tenir compte des règles précises de composition d'un code postal,
  mais bien seulement qu'il s'agit d'une suite de six caractères
  alternant entre une lettre et un chiffre. Les lettres peuvent être
  fournies en majuscule ou en minuscule et l'espace entre le troisième
  et le quatrième symbole peut être présent ou non.
  \begin{sol}
    \verb=[a-zA-z][0-9][a-zA-z]\s?[0-9][a-zA-z][0-9]= si l'on ne
    permet que zéro ou une espace entre les deux blocs. S'il n'y a pas
    de limite au nombre d'espaces, remplacer «\verb=?=» par
    «\verb=*=».
  \end{sol}
\end{exercice}

\begin{exercice}
  Écrire une commande \code{sed} permettant de changer le séparateur
  décimal dans les temps du fichier \code{100metres.data} pour une
  virgule.
  \begin{sol}
    \code{sed} est l'outil idéal pour de tels traitements simples à
    effectuer ligne par ligne.
    \begin{Schunk}
\begin{Verbatim}
$ sed 's/\./,/' 100metres.data
\end{Verbatim}
    \end{Schunk}
    Pour placer le fichier modifié dans, disons,
    \code{100metres-dec.data}, utiliser
    \begin{Schunk}
\begin{Verbatim}
$ sed 's/\./,/' 100metres.data > 100metres-dec.data
\end{Verbatim}
    \end{Schunk}
  \end{sol}
\end{exercice}

\begin{exercice}
  \begin{enumerate}
  \item Extraire du fichier \code{100metres.data} informations des
    temps réalisés au mois d'août.
  \item Extraire les informations des temps de moins de 10~secondes.
  \item Extraire les lignes du fichier satisfaisant les deux
    conditions ci-dessus. Vous pouvez utiliser l'opérateur de
    redirection \verb=|= de Unix (\autoref{sec:informatique:os:unix})
    pour combiner les deux commandes précédentes.
  \end{enumerate}
  \begin{sol}
    \begin{enumerate}
    \item C'est un travail pour \code{grep}. La solution la plus serait:
      \begin{Schunk}
\begin{Verbatim}
$ grep '-08-' 100metres.data
\end{Verbatim}
      \end{Schunk}
      Cependant, \code{grep} n'aime pas cette commande puisque le
      symbole \verb=-= est utilisé pour passer des options. Dans ce
      cas, il vaut mieux utiliser une option \verb=-e= pour déclarer
      explicitement à \code{grep} que ce qui suit est l'expression à
      rechercher:
      \begin{Schunk}
\begin{Verbatim}
$ grep -e '-08-' 100metres.data
\end{Verbatim}
      \end{Schunk}
      Autrement, simplement ajouter quelque chose à chercher avant le tiret:
      \begin{Schunk}
\begin{Verbatim}
$ grep '.*-08-' 100metres.data
\end{Verbatim}
      \end{Schunk}
    \item Le plus simple, ici, consiste à utiliser \code{awk} puisque
      les seconds champs --- les temps --- seront automatiquement
      disponibles.
      \begin{Schunk}
\begin{Verbatim}
$ awk '$2 < 10 { print }' 100metres.data
\end{Verbatim}
      \end{Schunk} %$
      Malheureusement, cette commande risque de ne pas fonctionner sur
      certains systèmes qui traitent la virgule comme le séparateur
      décimal, notamment les Mac en configuration française. Dans ce
      cas, essayer plutôt (j'ai supprimé la commande \verb={ print }=
      ci-dessous puisqu'elle est implicite):
      \begin{Schunk}
\begin{Verbatim}
$ LC_NUMERIC="en_US.UTF-8" awk '$2 < 10' 100metres.data
\end{Verbatim}
      \end{Schunk} %$
      Une solution avec \code{grep} consisterait à rechercher un
      «\verb=9.=» après l'espace sur chaque ligne:
      \begin{Schunk}
\begin{Verbatim}
$ grep ' 9\.' 100metres.data
\end{Verbatim}
      \end{Schunk}
    \item Nous pouvons combiner les deux commandes ainsi:
      \begin{Schunk}
\begin{Verbatim}
$ grep -e '-08-' 100metres.data | awk '$2 < 10'
\end{Verbatim}
      \end{Schunk} %$
      ou
      \begin{Schunk}
\begin{Verbatim}
$ grep -e '-08-' 100metres.data | grep ' 9\.'
\end{Verbatim}
      \end{Schunk}
    \end{enumerate}
  \end{sol}
\end{exercice}

\begin{exercice}
  Pour tous les appels à une fonction \code{f} comportant deux
  arguments dans le fichier \code{implementation.R} livré avec le
  présent document, changer le nom de la fonction pour \code{fun} et
  inverser l'ordre des deux arguments.
  \begin{sol}
    Ce traitement s'effectue bien avec \code{sed}:
    \begin{Schunk}
\begin{Verbatim}
$ sed -E 's/( |^)f\((.*), (.*)\)/fun(\3, \2)/' \
  implementation.R
\end{Verbatim}
    \end{Schunk}
    Le premier groupe \verb=( |^)= capture une espace avant \code{f}
    ou le début de la ligne.
  \end{sol}
\end{exercice}

\begin{exercice}
  Déterminer si la commande \code{print} est nécessaire ou non dans la
  toute dernière commande \icode{awk} de la
  \autoref{sec:texte:exemples}.
  \begin{sol}
    La commande \code{print} n'est pas nécessaire puisqu'il s'agit de
    l'action implicite dans \code{awk}.
  \end{sol}
\end{exercice}

\begin{exercice}
  Dans le code source du présent document\footnote{%
    Disponible en clonant avec Git le projet
    \url{https://projets.fsg.ulaval.ca/git/scm/vg/programmer-avec-r-develop.git}.}, %
  le fichier \code{docs/index.md} contient une ligne de ce type
  (affichée sur trois lignes ici pour des raisons de contraintes
  horizontales):
  \begin{Schunk}
\begin{Verbatim}
2017.11-2 ([notes de mise à jour]
  ({{ site.github.repository_url }}
  /releases/tag/v2017.11-2/))
\end{Verbatim}
  \end{Schunk}
  Le numéro de version du document apparait donc deux fois sur la
  ligne. Ce numéro de version est composé ainsi: année de publication;
  un point; mois de publication; une portion optionnelle composée d'un
  tiret, de un ou plusieurs chiffres et possiblement d'une lettre.
  Autrement dit, les chaines de caractères suivantes constituent
  toutes des numéros de versions valides: «2017.11», «2017.11-2»,
  «2017.11-2a», «2017.11-42k».

  Écrire une commande \code{sed} permettant de remplacer le numéro de
  version qui se trouverait actuellement dans le fichier
  \code{index.md}, quel qu'il fut, par le numéro de version
  2017.12-1a.
  \begin{sol}
    Invoquez \code{sed} avec l'option \code{-E} dans le cas présent
    pour utiliser les expressions régulières «étendues» ou modernes
    (l'opérateur \verb=?= n'existe pas dans la version de \code{sed}
    qui utilise les expressions de base sous macOS). À noter dans la
    réponse ci-dessous: l'expression régulière nécessaire et la
    présence de la commande \code{g} pour remplacer toutes les
    correspondances sur la ligne.
    \begin{Schunk}
\begin{Verbatim}
$ sed -E \
  's/[0-9]{4}\.[0-9]{2}(-[0-9]+[a-z]*)?/2017.12-1a/g' \
  index.md
\end{Verbatim}
    \end{Schunk}
    Le fichier \code{docs/Makefile} du projet contient une mise en
    œuvre avec \code{awk} équivalente, mais plus robuste.
  \end{sol}
\end{exercice}

\Closesolutionfile{solutions}

%%% Local Variables:
%%% mode: latex
%%% TeX-engine: xetex
%%% TeX-master: "programmer-avec-r"
%%% coding: utf-8
%%% End:
